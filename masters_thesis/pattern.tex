%
In this section, we will present our method for solving
formulas that contain implementations of multiplications.
%
Our method attempts to match multiplications that
are decomposed using long or Wallace tree multiplication.
%
If we match some subterms of the input formula
as instances of the
multiplications, we add tautologies stating that the terms are 
equal to the multiplications of the matched bit-vectors.
%
Our matching method may find multiple matches for a subterm.
%
We add a tautology for each match to the input and solve using
an available solver.
%
Let us first present our method of matching long multiplication. 

\subsection{Matching long multiplication}
\begin{algorithm}[t]
 \caption{\textsc{MatchLong}($t$)}
 \label{alg:long}
 \begin{algorithmic}[1]
   \Require $t$ : a term in $\qfbv$
   \Ensure $M$ : matched multiplications := $\emptyset$
   \If{ $t == (s_{1k_1} \concat ... \concat s_{11})+...+(s_{pk_p} \concat ... \concat s_{p1})$}
   \State Let $w$ be such that for some $s_{ij} = (0^{w} \concat a) * (0^{w} \concat b)$ and  $len(a) == w$ %:=  minimum bit length of $s_{ij}$
   \State $\Lambda := \lambda i. \emptyset$
   \For{ each $s_{ij}$ }
   \State $o := (\sum_{j' < j} len( s_{ij'}))/w$
   \IIf{ $s_{ij} == 0$ } {\bf continue;}
   \If{ $s_{ij} == (0^{w} \concat a) * (0^{w} \concat b)$  and $len(a) == w$} $\Lambda_o.insert( a * b )$
   \Else~\Return{$\emptyset$}
   \EndIf
   \EndFor
   \State \Return{\textsc{getMultOperands}($\Lambda$,$w$)}
   \EndIf
   % \Else
   \State \Return{$\emptyset$}
 \end{algorithmic}
\end{algorithm}  


%--------------------- DO NOT ERASE BELOW THIS LINE --------------------------

%%% Local Variables:
%%% mode: latex
%%% TeX-master: "main"
%%% End:


In Algorithm~\ref{alg:long}, we present a function $\textsc{MatchLong}$
that takes a $\qfbv$ term $t$ and returns a set of matched multiplications.
%
The algorithm and the subsequent algorithms are written such that as soon
as it becomes clear that no multiplication can be matched then
they return empty set. 
%
At line 1, we match $t$ with a sum of concatenations, and if the match fails
then clearly $t$ is not a long multiplication.
%
At line 2, we find a partial product among $s_{ij}$ and extract
the block size $w$ used by the long multiplication.
%
The loop at line $4$ populates the vector of the set of partial products $\Lambda$.
%
$\Lambda_i$ denotes the partial products that are aligned at the $i$th block.
%
Each $s_{ij}$ must either be $0$ or a partial product of the form mentioned in the
condition at line 7.
%
Otherwise, $t$ is declared unmatched at line 9. 
%
At line 5, we have computed the alignment $o$ for $s_{ij}$.
%
If $s_{ij}$ happens to be a partial product, it is inserted in
$\Lambda_o$ at line 8.
%
At line 10, we call $\textsc{getMultOperands}$ to identify the operands
of the long multiplication from $\Lambda$ if $t$ is indeed a long
multiplication.

\subsection{Partial products to operands}
\begin{algorithm}[t]
 \caption{\textsc{getMultOperandRec}($Cs$)}
 \label{alg:hb}
 \begin{algorithmic}[1]
   \Ensure $Cs$ : array of multisets of the partial products
   \State $m$ : candidate multiplier 
   \State $n$ : candidate multiplicand
   \State Choose largest $k$ such that $Cs_k\neq \emptyset$
   \If{$Cs_k == \{a*b\} $}
   \State $m_k := a; n_k := b$; $backtrack_k := \lfalse$; 
   \Else
   \State \Return{ FAILED}
   \EndIf
   \State $i := k$
   \While{ $i > 0$}
   \State $i = i - 1 $
   \State $C := Cs_i$
   \For{ $j \in (k-1)..(i+1)$}
   \If{$m_{j} \neq 0$ and $n_{k+i-j} \neq 0$}
   \IIf{ $m_j*m_{k+i-j} \not\in C$} {\bf goto }{\textsc{Backtrack}}
   \State $C := C - \{m_j*m_{k+i-j}\}$
   \EndIf
   \EndFor
   % \IIf{ $|C| > 2$} {\bf goto }{\textsc{Backtrack}}
   \State {\bf match} $C$ {\bf with}
   \State\quad $\mid$ $\{m_k*b,n_k*d\}$ $\rightarrow$ $m_i := d; n_i := b$;
   $backtrack_i := (m_k == n_k)$; 
   \State\quad $\mid$ $\{m_k*n_k\}$ $\rightarrow$ $m_i := 0; n_i := m_k;$
   $backtrack_i := \ltrue$; 
   \State \quad$\mid$ $\{m_k*b\}$ $\rightarrow$ $m_i := 0; n_i := b;$
   $backtrack_i := (m_k == n_k)$; 
   \State \quad $\mid$ $\{n_k*b\}$ $\rightarrow$ $m_i := b; n_i := 0;$\
   $backtrack_i := \lfalse$; 
   \State \quad $\mid$ $\{\}$ $\rightarrow$ $m_i := 0; n_i := 0;$\
   $backtrack_i := \lfalse$; 
   \State \quad $\mid$ $\_$ $\rightarrow$ {\bf goto }{\textsc{Backtrack}};
   \State {\bf continue;}
   \State \textsc{Backtrack:}
   \State \quad Choose smallest $i' > i$ such that $backtrack_i == \ltrue$
   \State \quad {\bf if} no $i'$ found {\bf then} \Return{ FAILED}
   \State \quad $i := i'$
   \State \quad \textsc{SWAP}($m_i,n_i$); $backtrack_{i} := \lfalse$
   \EndWhile
   \State Let $l$ be the smallest number such that $Cs_l\neq \emptyset$
   \State Choose $o \in 0..l$\ashu{More constraints over needed?!!}
   \State Right shift $m$ until $o$ trailing $0$ blocks in $m$
   \State Right shift $m$ until $l-o$ trailing $0$ blocks in $n$
   \State \Return $(m,n)$
 \end{algorithmic}
\end{algorithm}  

%--------------------- DO NOT ERASE BELOW THIS LINE --------------------------

%%% Local Variables:
%%% mode: latex
%%% TeX-master: "main"
%%% End:


In Algorithm~\ref{alg:operand}, we present a function
$\textsc{getMultOperands}$ that takes a vector of multiset of partial
products $\Lambda$ and block width $w$, and returns a set of matched
multiplications.
%
The algorithm proceeds by incrementally choosing a pair of operands with
insufficient information and backtracks if the guess is found to be
wrong.


At line 1, we compute $h$ and $l$ that establishes the range of the
search for the operands.
%
We maintain two candidate operands $x$ and $y$ of size $hw$.
%
We also maintain a vector of bits $backtrack$ that encodes 
the possibility of flipping the uncertain decisions.
%
Due to the scheme of the long multiplication, the highest
non-empty entry in $\Lambda$ must be a singleton set.
%
If $\Lambda_h$ contains a single partial product $a*b$,
we assign $x_h$ and $y_h$ the operands of $a*b$ arbitrarily.
%
We assign $\lfalse$ to $backtrack_h$, which states that
no need of backtracking at index $h$.
%
If $\Lambda_h$ does not contain a single partial product,
we declare the match has failed by returning $\emptyset$.
%
The loop at line 8 iterates over index $i$ from $h$ to $1$.
%
In each iteration, it assigns values to $x_i$, $y_i$, and $backtrack_i$. 
%

The algorithm may not have enough information at the $i$th iteration
and the chosen value for $x_i$ and $y_i$ may be wrong.
%
Whenever, the algorithm realizes that such a mistake has happened
it jumps to line 31.
%
It increases back the value of $i$ to the latest $i'$ that allows
backtracking.
%
It swaps the assigned values of $x_i$ and $y_i$, and disables future
backtracking to $i$ by setting $backtrack_i$ to
$\lfalse$.
%

Let us look at the loop at line 8 again.
%
We also have variables $l_x$ and $l_y$ that contain the index of
the least non-zero entries in $x$ and $y$, respectively.
%
At line 9, we decrement $i$ and $\Lambda_i$ is copied to $C$.
%
At index $i$, the sum of the aligned partial products is the following.
$$
x_{h}*y_{i} + \underbrace{x_{h-1}*y_{i+1} + \dots + x_{i+1}*y_{h-1}}_{\text{operands seen at the earlier iterations}} + x_{i}*y_{h}
$$
We have already chosen the operands of the middle partial products in
the previous iterations.
%
Only the partial products at the extreme ends have $y_i$ and $x_i$ that are
not assigned yet.
%
In the loop at line 10, we remove the middle partial products.
%
If any of the needed partial product is missing then we may have made a mistake
earlier and we jump for backtracking.
%
After the loop, we should be left with at most two partial products in $C$
corresponding to $x_{h}*y_{i}$ and $x_{i}*y_{h}$.
%
We match $C$ with the five patterns at lines 14-19 and
update $x_i$, $y_i$, and $backtrack_i$ accordingly.
%
If none of the pattern match, we jump for backtracking at line 20.
%
In some cases we clearly determine the value of $x_i$ and $y_i$, and
we are not certain in the other cases.
%
We set $backtrack_i$ to $\ltrue$ in the uncertain cases to indicate
that we may return back to index $i$ and swap $x_i$ and $y_i$.
%
In the following list, we discuss the uncertain cases.
%
\begin{itemize}
\item[line 15:] If $C$ has two elements $x_h*b$ and $y_h*d$,
there is an ambiguity in choosing $x_i$ and $y_i$
if $x_h == y_h$.
%
% Thus, if  $x_h == y_h$, we set $backtrack_i$ to $\ltrue$.
\item[line 16:] If $C$ has a single element $x_h*y_h$, there  
are two possibilities.
\item[line 17:] If $C = \{x_h*b\}$ and $b$ is not $y_h$ then 
  similar to the first case there is an ambiguity in
  choosing $x_i$ and $y_i$ if $x_h == y_h$. Line 18 is similar.
\item[line 19:] If $C$ is empty then there is no uncertainty. % $x_i = y_i = 0$.
\end{itemize}
%
At line 21-22, we update $l_x$ and $l_y$ appropriately.
%
The condition at line 23 ensures that the expected
least index $i$ such that $\Lambda_i \neq \emptyset$ is greater than 0.    
%
At line 24, we check if $i==1$, which means a match has been successful.
%
To find the appropriate operands, we need to right shift $x$ and $y$
such that the total number of their trailing zero blocks is $l-1$.
%
We add the matched $x*y$ to the match store $M$.
%
And, the algorithm proceeds for backtracking to find if more matchings
exist.

\subsection{Matching Wallace tree multiplication}
\begin{algorithm}[t]
 \caption{\textsc{RecognizeWallaceTree}($t$)}
 \label{alg:grade}
 \begin{algorithmic}[1]
   \Ensure $t$ : a term in $\qfbv$
   \If{ $t == (t_{k} \concat ... \concat t_{1})$}
   % \State $w$ :=  minimum bit length of $s_{ij}$
   \State $\Lambda := \lambda i. \emptyset$;
   $\Delta := \lambda i. \emptyset$
   \For{ $i \in 1..k$ }
   \State $S$ := $\{t_i\}$
   \For{ $S \neq \emptyset$} 
   \State $t \in S;$ $S := S - \{t\}$
   \If{ $t == s_1 \lxor .... \lxor s_p$}
   \State $S := S \union \{s_1,..,s_p\}$;$\Delta_i := \Delta_i \union \{s_1,..,s_p\}$
   \ElsIf{$t$ is carry output of $fullAdder(a,b,c)$ and $a,b,c \in \Delta_{i-1}$}
   \State $\Delta_i.insert( t )$
   \State $\Delta_{i-1} := \Delta_{i-1}- \{a,b,c\}$
   \ElsIf{$t$ is carry output of $halfAdder(a,b)$ and $a,b \in \Delta_{i-1}$}
   \State $\Delta_i.insert( t )$
   \State $\Delta_{i-1} := \Delta_{i-1}- \{a,b\}$
   \ElsIf{$t == a \land b$}
   \State $\Lambda_i.insert( a * b )$;
   \IIf{ $t \neq t_i$} $\Delta_i.insert( t )$
   \Else~\Return{$\emptyset$}
   \EndIf
   \EndFor
   \IIf{ $i > 1$ and $ \Delta_{i-1} \neq \emptyset $}~\Return{$\emptyset$}
   \EndFor
   \State \Return{\textsc{getMultOperand}($\Lambda$,$1$)}
   \EndIf
   % \Else
   \State~\Return{$\emptyset$}
 \end{algorithmic}
\end{algorithm}  


%--------------------- DO NOT ERASE BELOW THIS LINE --------------------------

%%% Local Variables:
%%% mode: latex
%%% TeX-master: "main"
%%% End:


A Wallace tree has a cascade of adders that take partial products 
and carry bits as input to produce the output bits.
%
In our matching algorithm, we find the set of inputs
to the adders for an output bit and classify them into
partial products and carry bits.
%
The half and full adders are defined as follows.
\begin{align*}
sumHalf(a,b) = a \lxor b  \qquad& \quad sumFull(a,b,c) = a \lxor b \lxor c\\
carryHalf(a,b) = a \land b \qquad&\quad
carryFull(a,b,c) = (a \land b) \lor (b \land c) \lor (c \land a)
\end{align*}
%
The sum output of a half/full adders are the result of 
xor operations of inputs.
%
To find the input to the cascaded adders, we start from an
output bit and follow backward until we find the input that
are not the result of some xor.

In Algorithm~\ref{alg:wallace}, we present a function
$\textsc{MatchWallace}$ that takes a $\qfbv$ term $t$ and returns a
set of matched multiplications.
%
At line 1, $t$ is matched with a concatenation of single bit terms $t_k$,..,$t_1$.
%
Similar to Algorithm~\ref{alg:long},
we maintain the partial product store $\Lambda$.
%
For each $i$,
we also maintain the multiset of terms $\Delta_i$ that were used as 
inputs to the adders for the $i$th bit.
%
In the loop at line 6, we traverse down the subterms until
a subterm is not the result of a xor.
%
In the traversal, we also collect the inputs of the visited xors
in $\Delta_i$, which
will help us in checking that all the carry inputs in adders for $t_{i+1}$
are generated by the adders for $t_i$.
%
If the term $t$ is not the result of xors then
we have the following possibilities.
%
\begin{itemize}
\item[line 10-13:]
  If $t$ is the carry bit of a half/full adder, and the inputs, the intermediate
  result of the sum bit, and the output sum bit
  of the adder are in $\Delta_{i-1}$ then we remove the inputs and intermediate result
  of the adder from $\Delta_{i-1}$.
  We do not remove the output sum bit from $\Delta_{i-1}$, since it may be
  used as input to some other adder.
\item[line 14-15:] If $t$ is a partial product, we record it in $\Lambda_i$.
\item[line 16:] Otherwise, we return $\emptyset$.
\end{itemize}
At line 17, we check that $\Delta_{i-1} = \{t_{i-1}\}$, i.e., all carry bits from
the adders for $t_{i-1}$ are consumed by the adders for $t_i$ exactly once.
%
Again if the check fails, we return $\emptyset$.
%
After the loop at line 3, we have collected the partial products in $\Lambda$.
%
At line 18, we call $\textsc{getMultOperands}(\Lambda,1)$ to get all
the matching multiplications.

\subsection{Our solver}
\begin{algorithm}[t]
 \caption{\textsc{OurSolver}($F$)}
 \label{alg:solver}
 \begin{algorithmic}[1]
   \Ensure $F$ : a $\qfbv$ formula
   \State \textsc{SMTSolver}.add($F$)
   \For{ each subterm $t$ in $F$}
   \If{ $M$ := \textsc{RecognizeGradeSchoolMultiplier}($t$) }
   \For{ each $m*n \in M$}
   \State \textsc{SMTSolver}.add($m*n = t$)
   \EndFor
   \EndIf
   \EndFor
   \State \Return{ \textsc{SMTSolver}.checkSat() }
 \end{algorithmic}
\end{algorithm}  

%--------------------- DO NOT ERASE BELOW THIS LINE --------------------------

%%% Local Variables:
%%% mode: latex
%%% TeX-master: "main"
%%% End:


Using the above pattern matching algorithms, we modify an existing
solver $\textsc{SMTSolver}$, which
is presented in Algorithm~\ref{alg:solver}.
%
$\textsc{OurSolver}$ adds the input formula $F$ in $\textsc{SMTSolver}$.
%
For every subterm of $F$, we attempt to match with both long multiplication
or Wallace tree multiplication.
%
For each discovered matching $x*y$, we add a bit-vector tautology $[x*y = t]$ 
to the solvers, which is obtained after
appropriately zero-padding $x$, $y$, and $t$.
%
% \ashu{disjunctive assertion of the tautologies} 

% \ashu{Proof generation: Are we doing this??}

%--------------------- DO NOT ERASE BELOW THIS LINE --------------------------

%%% Local Variables:
%%% mode: latex
%%% TeX-master: "main"
%%% End:
