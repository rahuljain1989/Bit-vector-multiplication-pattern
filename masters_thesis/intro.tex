% Bit-vector is important
%
In recent years, SMT (Satisfiability Modulo Theories) solving has
emerged as a powerful technique for testing, analysis and verification
of hardware and software systems.  A wide variety of tools today use
SMT solvers as part of their core reasoning engines; examples include
bounded model checkers~\cite{hwcbmc,boolector,ebmc,cbmc}, static
assertion checkers~\cite{corral,boogie}, word-level symbolic
trajectory evaluators~\cite{wste}, constrained test
generators~\cite{crv1,crv2,dart}, concolic simulators~\cite{concolic}, among
others.  A common approach used by these tools is to model the
behaviour of a system using formulas in a combination of first-order
theories, and reduce the given problem to one of checking the
(un)satisfiability of a formula in the combined theory.  SMT solvers
play a central role in this scheme of things, since they combine
decision procedures of individual first-order theories to check the
satisfiability of formulas in the combined theory.  Techniques to
improve the performance of SMT solvers are therefore important, and
have been widely researched. In this paper, we add to the repertoire
of such techniques by presenting a pre-processing heuristic that is
particularly useful when reasoning about formulas in the
quantifier-free theory of fixed-width bit-vectors with multiplication.
Our experiments show that three state-of-the-art SMT solvers --
{\zthree}~\cite{zthree}, {\cvcfour}~\cite{cvcfour} and
{\boolector}~\cite{boolector} -- benefit significantly from this
technique on several problem instances, especially when the input
formula is unsatisfiable.

%% Theories commonly
%% supported in modern SMT solvers include the quantifier-free theories
%% of fixed-width bit-vectors, arrays, lists, strings, among
%% others~\cite{kroening-book,smtlibv2}.
%% Since reasoning in these individual theories is often much more
%% efficient than reasoning about the bit-level representation of the
%% corresponding data types, techniques based on SMT solving hold a lot
%% of promise as far as scaling to large applications is concerned.  The
%% impressive progress made in SMT solving over the last two
%% decades~\cite{smtprogress} has also substantially lived up to this
%% hope. 

The motivation for our work comes primarily from word-level bounded
model checking~\cite{cbmc,hwcbmc} and symbolic trajectory
evaluation~\cite{wste} of embedded hardware systems.  Specifically, we
focus on systems that process data, represented as fixed-width
bit-vectors, using arithmetic operations.  Examples of such systems
include digital signal processing filters, graphics accelerators,
encryption and decryption modules, custom datapath implementations
etc.  When reasoning about such systems, it is often necessary to
check whether a high-level property, specified using bit-vector
arithmetic operators (viz. addition, multiplication, division), is
satisfied by a model of the system that implements some
data-processing algorithm.  For reasons related to performance, power,
area, ease of design etc., complex arithmetic operators with large
bit-widths are rarely implemented as monolithic combinational blocks
in embedded systems.  For example, a $128$-bit multiplier is unlikely
to be implemented as a single monolithic block.  Instead, its operands
are likely to be decomposed into smaller, say $8$-bit, blocks, and the
overall product obtained by combining these blocks using one of
several multiplication algorithms, viz. long multiplication (also
called grade-school multiplication), Booth-encoded multiplication or
Wallace-tree multiplication.  Therefore, the SMT formula obtained from
word-level bounded model checking/symbolic trajectory evaluation is
likely to contain terms that refer to the same bit-vector arithmetic
operators, but represented in two widely differing ways.  For example,
the formula may have terms with the bit-vector multiplication operator
(derived from high-level specifications), and also terms that encode
the implementation of the operator using a Wallace-tree multiplier.
Efficiently reasoning about such a formula requires knowledge of the
semantic equivalence of the two representations.  Unfortunately, our
experiments show that three state-of-the-art SMT solvers (Z3, CVC4 and
Boolector) are unable to determine this semantic equivalence
efficiently.  Consequently, they end up \emph{bit-blasting} the
formula, effectively reasoning about a propositional encoding of both
representations.  While this works reasonably well in cases where the
SMT formula is satisfiable, state-of-the-art solvers encounter serious
scalability issues with this approach when reasoning about
unsatisfiable formulas.  This motivates us to ask \emph{if we can
  preprocess an SMT formula efficiently and provide ``hints'' to the
  SMT solver to help check (un)satisfiability efficiently.} We answer
this question positively in this paper.


At first sight, the above problem appears to be one of
reverse-engineering a decomposed implementation (encoded in the SMT
formula obtained from the system implementation) of a complex
word-level operation.  Indeed, this has been done by previous
researchers by matching a sub-formula with a pre-specified
``pattern''~\cite{earlier-pat-match-synopsys}.  A closer inspection,
however, reveals that there are several subtle complications. First,
and contrary to intuition, the same decomposed implementation may
correspond to multiple reverse-engineered bit-vector operations.
Second, even if we are able to recover a complex word-level operation
from a sub-formula corresponding to a decomposition in the given SMT
formula, rewriting the sub-formula with the recovered operation may
not correlate with improved performance when checking the
(un)satisfiability check of the overall SMT formula.  This can happen
for various reasons:
\begin{itemize}
\item When multiple word-level operations can be recovered from the
  same sub-formula, one cannot decide a priori which word-level
  operation will be beneficial for the overall satisfiability check.
\item The recovery is a local ``peep-hole'' operation that looks only
  at the sub-formula matching a specific ``pattern''.  Specifically,
  the recovery is oblivious of the context in which the sub-formula
  appears in the overal SMT formula, and rewriting the sub-formula
  with the recovered word-level operation may not help (and may even
  hurt) the overall (un)satisfiability check.  
\item In general, the sub-formula matching a pre-specified ``pattern''
  may help in simplifying some part of the SMT formula, while the
  recovered word-level operation may help in simplifying some other
  part of the SMT formula.  Re-writing the matched sub-formula with
  the recovered operation therefore precludes using the sub-formula
  to simplify the SMT formula.
\end{itemize}
Thus, naively replacing a pattern corresponding to a decomposed
implementation with the bit-level operator is unlikely to help SMT
solving in any but the simplest of cases.

In this paper, we present a pre-processing heuristic to address the
above problem, focusing only on bit-vector multiplication.  Given a
bit-vector formula $\varphi$ that contains terms with two different
representations of bit-vector multiplication, our heuristic searches
for patterns corresponding to two multiplication algorithms (long
multiplication and Wallace-tree multiplication) in the terms. Instead
of rewriting the matched sub-terms directly, as in earlier work, we
conjoin the formula $\varphi$ with additional assertions that equate a
matched sub-term with the corresponding bit-vector operator.  Since no
re-writes are done, this allows us to express multiple equivalences in
a convenient way.  Our experiments show that the added assertions
succeed in preventing bit-blasting in several cases, while in other
cases they significantly help in pruning the search space even after
bit-blasting.  Both benefits eventually translate to improved
performance of the SMT solver.  Since our heuristic simply
pre-processes the input formula by adding assertions, it is
independent of the internals of any specific SMT solver, and can be
combined with multiple solvers.  Indeed.  our experiments show that
the performance of different SMT solvers on the pre-processed formula
can vary.  Hence, we propose a portfolio approach to solving the
pre-processed formulas, and show experimentally that a portfolio
solver using pre-processed formulas significantly outperforms a
portfolio solver using the original formulas.



%% Many hardware and software verification problems are translated to the
%% satisfiability of quantifier free bit-vector(QF\_BV)
%% formulas~\cite{hardware,cbmc,more}.
%


%--------------------- DO NOT ERASE BELOW THIS LINE --------------------------

%%% Local Variables: 
%%% mode: latex
%%% TeX-master: "main"
%%% End: 
