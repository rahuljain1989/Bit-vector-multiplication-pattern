In this section, we present some basics of the theory of
quantifier-free fixed-width bit-vector formulas (\qfbv), briefly describe half and full adders, and discuss
two well-known multiplication algorithms of interest.

%

\section{Theory of Bit-vectors}

All computing systems operate by manipulating finite sequences of zeros and ones - bit-vectors. A computer system uses
bit-vectors to encode information, for example numbers. Owing to the finite
domain of these bit-vectors, the semantics of operations such as addition no
longer matches what we are used to when reasoning about unbounded types,
for example the natural numbers. As a simple example, consider the following formula:

\[( x-y > 0) \leftrightarrow (x > y)
\]

\noindent This formula holds over the integers, but if x and y are interpreted as finite-width bit vectors, this equivalence
no longer holds, owing to possible overflow of the subtraction operation.

A bit-vector is a vector of Boolean values with a given length $l$:
\[
b:\{0,. . .,l-1\} \rightarrow \{0,1\}
\]

%
\noindent We denote bit-vectors by $x$,$y$,$z$, etc., and often
%
refer to blocks of bits in a bit-vector.
%
For example, we may declare that a bit-vector $x$ is accessed in
blocks of width $w$.
%
Let $x_i$ denote the $i$th block of bits, with the block containing
the least significant bit (LSB) having index $0$.
%
% Similar notation is used for the vectors of any object.

\subsection{Syntax}

%A $\qfbv$~term $t$ and formula $F$ is constructed using
%the following grammar
%\begin{align*}
%t ::= &~ t * t \mid t + t \mid x \mid n^w \mid t \concat t  ....\\
%F ::= &~ t = t \mid t \bowtie t \mid \lnot F \mid F \lor F \mid F \land F %\mid F \lxor F \mid ... 
%\end{align*}

Grammar for a subset of bit-vector arithmetic:

\vspace{0.2 cm}

$formula ::= formula \land formula \mid \lnot formula \mid (formula) \mid atom$

$atom ::= term$ $rel$ $term \mid Boolean$ $Identifier \mid term[constant]$ 

$rel ::= $ $< \mid =$

$term ::= term$ $op$ $term \mid x \mid \sim term \mid n^w \mid term[constant:constant] \mid$

$\qquad \qquad ite(atom, term, term)$

$op ::= + \mid - \mid \cdot \mid / \mid << \mid >> \mid \& \mid $ $\mid$ $\mid \otimes \mid \concat$

\vspace{0.2 cm}

\noindent where the operators have their usual meaning, $x$ is a bit-vector variable, $n^w$ is a binary constant
represented using $w$ bits.
%
Note that we have only presented above parts of the grammar
that are relevant to our discussion.  For more details,
the reader is referred to~\cite{Kroeningbook,barrett}.
%
We assume that all variables and arithmetic operators are unsigned.
Following the SMTLIB~\cite{SMTLIB} convention, we also assume that
arguments and results of an arithmetic operator have the same bit width.
%
Let $len(t)$ denote the bit width of a term $t$.
%
If $w \geq len(t)$,
let $zeroExt(t,w)$ be a shorthand for  $0^{w-len(t)}\concat t$.

If an operator $\op$ is commutative, when matching patterns, we will
not make a distinction between $a \op b$ and $b \op a$.
%
We use the notation ``$t == s$'' to denote that $t$ and $s$ are
syntactically identical.
%
Given bit-vector terms $x$, $y$, and $t$, suppose $w = max(len(x),len(y),
len(t))$.
%
We use ``$[x*y = t]$'' to denote term $x'*y'=t'$, where $x' =
zeroExt(x, w)$, $y' = zeroExt(y, w)$, and $t' = zeroExt(t, w)$.
%
Similarly, the notation $[x*y]$ is used to denote $x' * y'$, where $x'
= zeroExt(x, len(x)+len(y))$ and $y' = zeroExt(y, len(x) + len(y))$.

%% \section{SMT solvers for \qfbv}

%% SMT(satisfiability modulo theory)
%% solvers are specialized solvers that solve 
%% formulas of a given theory.
%

\subsection{Signature}

A typical signature $\Sigma_{bv}$ for selected functions and predicates of the bit-vector theory is presented below. The set of sorts $\Sigma_{bv}^S$ contains
infinitely many sort symbols $[n]$ where n is a strictly positive natural number. Formally, $\Sigma_{bv}^S = \{[n] \mid n > 0 $ and $ n \in \mathcal{Z}\}$.
\begin{table}[h]
\centering
\caption{$\Sigma_{bv}$}
\begin{tabular}{c | l | c | l }
Symbol & Name & Arity & Sort Map $\tau$  \\
\hline \hline
$[0]_{[1]}$ & constant & 0 & $[1]$  \\
$[01]_{[2]}$ & constant & 0 & $[2]$  \\ 
$\concat$ & concat & 2 & $([n],[m],[n+m])$ \\
$[i:j]$ & extract & 1 & $([n],[j-i+1])$ for $n > i > j \ge 0$
 \\ 
$\sim$ & bitwise not & 1 & $([n],[n])$  \\
$\mid$ & bitwise or & 2 & $([n],[n],[n])$ \\
$+$ & plus & 2 & $([n],[n],[n])$ \\
$>>$ & right shift & 2 & $([n],[n],[n])$\\
$=_{[n]}$ & equal & 2 & $([n],[n],Bool)$  \\
$<$ & less than & 2 & $([n],[n],Bool)$ \\

\end{tabular}
\end{table}

\subsection{Semantics}

We now give a formal definition of the semantics of a fixed width bit-vector arithmetic formula. To help us define the semantics we first define three mappings:
\begin{itemize}
\item $bitToNat:[1] \rightarrow [0,1]$ defined as:
bitToNat(b) = if $b = 1$ then $1$ else $0$.
\item $natBv_n : \mathbb{N} \rightarrow [n]$ defined as: \\
$natBv_n(x) = \lambda i : [0, n-1].if (x \div 2^i ) $mod $2$ = $0$ then $0$ else $1$
\item $bvNat : [n]\rightarrow[0, 2^n-1]$ defined as: 
$bvNat(f) = \sum^{n-1}_{i=0} bitToNat(f(i)).2^i$
\end{itemize}

\noindent $bitToNat$ converts one bit bit-vector to the corresponding natural number 0 or 1. $natBv_n$ converts natural numbers to their corresponding bit-vectors and $bvNat$ converts bit-vectors to their corresponding natural numbers. With these three mappings in place, the semantics can be given by:

\begin{itemize}

\item For $[n] \in \Sigma_{bv}^S$, $A_{[n]} = \{\lambda x : [0, n-1].f (x) \mid f : [0, n-1] \rightarrow \{0, 1\}\}$
\item For constant symbols:
\begin{itemize}
\item  $\llbracket 0_{[1]}\rrbracket = \lambda x : [0].0$
\item $\llbracket 1_{[1]}\rrbracket = \lambda x : [0].1$
\end{itemize}
\item For function symbols:
\begin{itemize}

\item $\llbracket a_{[n]} \concat b_{[m]}\rrbracket = \lambda x : [0, n+m-1].$if $x < m$ then $\llbracket b\rrbracket(x)$ else $\llbracket a\rrbracket(x-m)$

\item $\llbracket a_{[n]} [i:j]\rrbracket = \lambda x : [0,i-j]. \llbracket a\rrbracket(j+x)$
\item $\llbracket \sim a_{[n]}\rrbracket
= \lambda x : [0, n-1].$ if $\llbracket a\rrbracket(x)=0$ then $1$ else $0$

\item $\llbracket a_{[n]} \mid b_{[n]}\rrbracket = \lambda x : [0, n-1].$ if $\llbracket a\rrbracket(x) =1$ or $\llbracket b\rrbracket(x)=1$ then $1$ else $0$

\item $ \llbracket a_{[n]} + b_{[n]}\rrbracket = natBv_n (bvNat(\llbracket a \rrbracket) + bvNat(\llbracket b \rrbracket))$

\item $\llbracket a_{[n]} >> b_{[n]}\rrbracket = natBv_n (bvNat(\llbracket a \rrbracket) \div 2^{bvNat(\llbracket b \rrbracket)})$

\end{itemize}

\item For predicate symbols:
\begin{itemize}
\item $\llbracket a_{[n]} =_{[n]} b_{[n]}\rrbracket =$ if $(bvNat(\llbracket a \rrbracket) = (bvNat(\llbracket b \rrbracket)$ then $true$ else $false$
\item $\llbracket a_{[n]} < b_{[n]}\rrbracket =$ if $(bvNat(\llbracket a \rrbracket) < (bvNat(\llbracket b \rrbracket)$ then $true$ else $false$
\end{itemize}
\end{itemize}

The
concatenation and extract operations are the only ones that can alter the
length of bit-vectors. The arithmetic operations interpret the bit-vector as a base-2 natural number and are
equivalent to the corresponding mod $2^n$ arithmetic operations, where n is the size of the bit-vectors.

\subsection{Decision procedures for bit-vectors}

The leading SMT solvers for \qfbv~apply several simplification and
rewriting passes to decide the satisfiability of the input formula.
If these do not succeed in solving the problem, the solvers bit-blast
the formula, i.e., translate the bit-vector formula to an
equivalent propositional formula on the constituent bits of the
bit-vectors.  This reduces the bit-vector satisfiability problem to
one of propositional satisfiability (SAT).
%
The bit-blasted SAT problem is then solved using conflict driven clause
learning (CDCL)\cite{cdcl1,cdcl2} based SAT procedures.
%

Below we present a generic bit-blasting algorithm to decide the satisfiability of the input formula $\phi$ and also define some terminology used to to present the same.

\begin{itemize}

\item $At(\phi)\rightarrow$ set of atoms in $\phi$ 
\item $a \in At(\phi)$ is replaced by $e(a) \rightarrow$ the propositional encoder of $a$ 
\item Propositional skeleton $e(\phi) \rightarrow$ resulting formula after propositional encoding of the atoms 
\item $T(\phi)\rightarrow$ set of terms in $\phi$ 
\item $e(t)\rightarrow$ a vector of new Boolean variables for each BV term $t$ $\in T(\phi)$ 
\item $e(t)_i\rightarrow$ variable for the bit with index $i$ of the term $t$ 
\end{itemize}

\begin{algorithm}[t]
 \caption{\textsc{Bit Blasting}}
  \label{alg:bitblast}
 \begin{algorithmic}[1]
   \State B := e($\phi$);
   \For{ $t_{[l]} \in T (\phi)$}
   \For{ $i \in \{0,\dots, l-1\}$}
   \State set $e(t)_i$ to a new Boolean variable;
   \EndFor
   \EndFor
   \For{ $a \in At(\phi)$}
   \State $B:=B \land Constraint(e, a);$
   \EndFor
   \For{ $t_{[l]} \in T(\phi)$}
   \State $B:=B \land Constraint(e,t);$ 
   \EndFor
   \State \Return{$B$}
 \end{algorithmic}
\end{algorithm}  



Constraint needed for a particular atom $a$ or term $t$ depends
on the atom or term, respectively. In the case of a bit-vector or a Boolean
variable, no constraint is needed, and $Constraint$ returns true. If $t$ is
a bit-vector constant $C_{[l]}$, $Constraint$ function generates: $\bigwedge^{l-1}_{i=0}(C_i \leftrightarrow e(t)_i)$. Otherwise, $t$ must contain a bit-vector operator. The constraint that is needed
depends on this operator. The constraints for the bitwise operators are
straightforward. As an example, consider bitwise $OR$, and let $t = a \mid_{[l]} b$, $Constraint$ function generates: $\bigwedge^{l-1}_{i=0}((a_i \lor b_i) \leftrightarrow e(t)_i)$. The constraints for the other bitwise operators follow the same pattern.

For our work, we assume access to a generic $\qfbv$~SMT solver, called
$\textsc{SMTSolver}$, with the standard interface.
%
We assume that this interface includes a function $add(F)$ that adds a
formula $F$ to the context of the solver, and a function $checkSat()$
that checks the satisfiability of the conjunction of all formulas
added to the context of the solver.  Note that such interface is
available with most state-of-the-art solvers, viz. {\boolector},
{\cvcfour} and {\zthree}.


\section{Adders}

An adder is a digital circuit that performs addition of numbers.
Although adders can be constructed for many numerical representations, such as binary-coded decimal or excess-3, the most common adders operate on binary numbers. We describe the two most commonly used 
adders, namely half adder and full adder.

\subsection{Half adder}

The half adder adds two single binary digits a and b. It has two outputs, $\sumOfHalfAdder$ and $\carryOfHalfAdder$. The $\sumOfHalfAdder$ signal represents the value of the addition for the current digit. The $\carryOfHalfAdder$ signal represents an overflow into the next digit of a multi-digit addition.
The simplest half adder design, incorporates an XOR gate for $\sumOfHalfAdder$ and an AND gate for $\carryOfHalfAdder$.

\[
\sumOfHalfAdder = a \oplus b \hspace{3 cm} \carryOfHalfAdder = a \land b 
\]

\begin{figure}[h]
\usetikzlibrary[arrows]
\centering
\begin{tikzpicture}[circuit logic US, minimum height=10mm]
\matrix[column sep=10mm]
{
\node (i0) {$a$}; & & \\
& \node [xor gate] (xor1) {}; & \node (o1) {$\sumOfHalfAdder$};\\
\node (i1) {$b$}; & \node[and gate] (and1) {}; & \node (o2) {$\carryOfFullAdder$};\\
};
 
\draw (i0.east) -- ++(right:3mm) |- (xor1.input 1);
\draw (i1.east) -- ++(right:3mm) |- (xor1.input 2);
\draw (xor1.output) -- ++(right:3mm) |- (o1.west);

\draw[*-] (xor1.input 1) ++(-7mm,0.5mm) |- (and1.input 1);
\draw[*-] (xor1.input 2) ++(-4mm,0.5mm) |- (and1.input 2);

\draw (and1.output) -- ++(right:3mm) |- (o2.west);
\end{tikzpicture}
\caption{Half adder circuit}
\end{figure}

\subsection{Full adder}

The full adder adds three single binary digits a, b and c. It also has two outputs, $\sumOfFullAdder$ and $\carryOfFullAdder$. Both the outputs have the same meaning as that for a half adder, the only difference being that full adder adds three digits. The simplest full adder design incorporates two XOR gate for $\sumOfFullAdder$ and, three AND and 2 OR gates for $\carryOfFullAdder$. 

\[
\sumOfFullAdder = a \oplus b \oplus c \hspace{3 cm} \carryOfFullAdder = (a \land b) \lor (a \land c) \lor (b \land c) 
\]

\begin{figure}[h]
\usetikzlibrary[arrows]
\centering
\begin{tikzpicture}[circuit logic US, minimum height=10mm]
\matrix[column sep=10mm]
{
\node (i0) {$a$}; & & & & & \\
& \node [xor gate] (xor1) {}; & \node[xor gate] (xor2) {}; & & &\\
\node (i1) {$b$}; & & \node (cin) {}; & & \node (o1) {$\sumOfFullAdder$}; &\\
& & \node[and gate] (and1) {}; & & &\\
\node (i2) {$c$}; & & \node[and gate] (and2) {}; & \node[or gate] (or1) {};& &\\
& & \node[and gate] (and3) {}; & & \node[or gate] (or2) {}; & \node (o2) {$\carryOfFullAdder$};\\
};
 
\draw (i0.east) -- ++(right:3mm) |- (xor1.input 1);
\draw (i1.east) -- ++(right:3mm) |- (xor1.input 2);
\draw (xor1.output) -- ++(right:3mm) |- (xor2.input 1);
\draw (xor2.output) -- ++(right:3mm) |- (o1.west);

\draw[*-] (xor1.input 1) ++(-7mm,0.5mm) |- (and1.input 1);
\draw[*-] (xor1.input 2) ++(-4mm,0.5mm) |- (and1.input 2);

\draw[*-] (and1.input 2) ++(-7mm,0.5mm) |- (and2.input 1);
\draw (i2.east) -- ++(right:3mm) |- (and2.input 2);
\draw[*-] (and2.input 2) ++(-10mm,-0.5mm) |- (xor2.input 2);

\draw[*-] (and1.input 1) ++(-15mm,0.5mm) |- (and3.input 1);
\draw (i2.east)  |- (and3.input 2);

\draw (and1.output)  -- ++(right:3mm) |- (or1.input 1);
\draw (and2.output)  -- ++(right:3mm) |- (or1.input 2);

\draw (or1.output)  -- ++(right:3mm) |- (or2.input 1);
\draw (and3.output)  -- ++(right:3mm) |- (or2.input 2);
\draw (or2.output) -- ++(right:3mm) |- (o2.west);


\end{tikzpicture}
\caption{Full adder circuit}
\end{figure}

\section{Multipliers}

Multiplication is an expensive operation to implement in hardware.
%
There are several designs of multipliers for varying
resource constraints.
%
If one can have a large number of gates then Wallace tree
multiplier can be used.
%
Otherwise, one may decompose the multiplication task in
small multiplications and combine the results appropriately.
%
For example, long multiplier and Booth multiplier.
%
Here, we will discuss long and Wallace tree multiplier.

\subsection{Long multiplier}\label{sec:long-mult}

Consider bit-vectors $x$ and $y$ that are partitioned into $k$ blocks of width $w$ bits each. Thus the total width of each bit-vector is $k \cdot w$. 
% \ashu{Rahul: suggest a rewrite.}
The long multiplier decomposes the multiplication of two $(k\cdot w)$-bit wide bit-vectors into $k^2$ multiplications of $w$-bit wide bit-vectors. The corresponding $k^2$ products, called {\em partial products}, are then added with appropriate left-shifts to obtain the final
result. 
%
%The partial products are summed with appropriate offsets to obtain
%the final result.
%
The following notation is typically used to illustrate
long multiplication.
%
\begin{center}
\begin{tabular}{c@{\quad}c@{\quad}c@{\quad}c@{\quad}c@{\quad}c@{\quad}c}
  &&& $x_{k}$ & ... & $x_1$&\\ 
  &&& $y_{k}$ & ... & $y_1$&$*$\\ \hline\vspace{-2pt}
  &&&$x_k*y_1$& ... & $x_1*y_1$&\\\vspace{-2pt}
  &&$\iddots$&$\vdots$& $\iddots$ && \\
  &$x_k*y_k$& ... &$x_1*y_k$&  & +&\\\hline
\end{tabular}  
\end{center}
Here, the $x_i*y_j$s are the partial products. The partial product $x_i*y_j$ is left shifted $(i+j-2) \cdot w$ bits before being added. In the above representation, all partial
products that are left-shifted by the same amount are aligned in a single column.
After the left shifts, all the partial results are added in some order. 
%
%In the above scheme all the partial products that have same offset are 
%aligned in single column.
%
%After the shifts, all the partial results are added in some order.
%
Note that the bit-width of each partial product is $2 \cdot w$.
%
Since the syntax of $\qfbv$ requires the bit-widths of the arguments
and result of the $*$ operator to be the same, we denote the partial
product $x_i * y_j$ as $(0^w \concat x_i)*(0^w \concat y_j)$ for our
purposes. Note further that the bits of the partial products in
neighbouring columns (in the above representation of long
multiplication) overlap; hence the sums of the various columns can not
be simply concatenated. The long multiplication algorithm does not
specify the order of addition of the shifted partial
products. Therefore, there are several possible implementations for a
given $k$ and $w$.

\begin{example}
  Consider bit-vectors $v_3,v_2,v_1$, $u_3,u_2$, and $u_1$, each of
  bit-width 2.  Let us apply long multiplication on $v_3 \concat v_2
  \concat v_1$ and $u_3 \concat u_2 \concat u_1$.  We obtain the
  following partial products.
\begin{center}
\vspace{-2ex}
\begin{tabular}{c@{\quad}c@{\quad}c@{\quad}c@{\quad}c@{\quad}c@{\quad}c}
  &&& $v_3$ & $v_2$ & $v_1$&\\ 
  &&& $u_3$ & $u_2$ & $u_1$&$*$\\ \hline
  &&& $v_3*u_1$ & $v_2*u_1$ & $v_1*u_1$&\\
  && $v_3*u_2$ & $v_2*u_2$ & $v_1*u_2$ && \\
  & $v_3*u_3$ & $v_2*u_3$ &$v_1*u_3$&  & +&\\\hline
\end{tabular}
% \vspace{-1ex}
\end{center}
The following term is one (of several) possible combinations of the
partial products using concatenations and summations to obtain the
final product.
\vspace{-1ex}
\begin{align*}
  ((v_3*u_3) \concat (v_3*u_1) \concat (v_1*u_1)) +
  (0^2 \concat (v_2*u_3) \concat (v_2*u_1) \concat 0^2) +\\
  (0^2 \concat (v_3*u_2) \concat (v_1*u_2) \concat 0^2) +
  (0^4 \concat (v_2*u_2) \concat 0^4) + (0^4 \concat (v_1*u_3) \concat 0^4)
\vspace{-1ex}
\end{align*}
%
%Each partial product $x_i * y_j$ is 4 bit wide.
%
Note that we did not concatenate two partial products that appear next
to each other in the tabular representation, because their bits can
potentially overlap.

\end{example}


\begin{example}
  Consider bit-vectors $v_1,v_2,u_1$, and $u_2$, each of bit-width 2.
  Let us apply long multiplication on
  $v_2 \concat 0^2 \concat v_1$ and $u_2 \concat v_2 \concat u_1$.
  We obtain the following partial products.
\begin{center}
  \vspace{-2ex}
\begin{tabular}{c@{\quad}c@{\quad}c@{\quad}c@{\quad}c@{\quad}c@{\quad}c}
  &&& $v_2$ & $0^2$ & $v_1$&\\ 
  &&& $u_2$ & $v_2$ & $u_1$&$*$\\ \hline
  &&&$v_2*u_1$& $0^4$ & $v_1*u_1$&\\
  &&$v_2*v_2$&$0^4$& $v_1*v_2$ && \\
  &$v_2*u_2$& $0^4$ &$v_1*u_2$&  & +&\\\hline
\end{tabular}
\end{center}
Note that while adding the shifted partial products, if the non-zero
bits of a subset of shifted partial products do not overlap, then we
can simply concatenate them to obtain their sum. Finally, we can sum
the concatenated vectors thus obtained to calculate the overall
product. The following is one possible combination of 
concatenations and summations for the long multiplication in this
example.
%
\vspace{-1ex}
$$
( 0^4 \concat (v_1*u_2) \concat (v_1*u_1)) +
((v_2*u_2) \concat (v_2*u_1) \concat 0^4) +
(0^2 \concat (v_2*v_2) \concat (v_1*v_2) \concat 0^2)
\vspace{-1ex}
$$
\end{example}

\begin{example}

  As another interesting example, consider long multiplication applied to 
  $v_2 \concat 0^2 \concat v_2$ and $0^2 \concat v_1 \concat v_1$, where
  $v_1$ and $v_2$ have bit-width $2$.
  We obtain the following partial products.
\begin{center}
\begin{tabular}{c@{\quad}c@{\quad}c@{\quad}c@{\quad}c@{\quad}c@{\quad}c}
  &&& $v_2$ & $0^2$ & $v_2$&\\ 
  &&& $0^2$ & $v_1$ & $v_1$&$*$\\ \hline
  &&&$v_1*v_2$& $0^4$ & $v_1*v_2$&\\
  &&$v_1*v_2$&$0^4$& $v_1*v_2$ &+&\\\hline
  %&$v_2*u_2$& $0^4$ &$v_1*u_2$&  & +&\\\hline
\end{tabular}
\end{center}
Note that, if we had applied long multiplication to $v_1 \concat 0^2
\concat v_1$ and $0^2 \concat v_2 \concat v_2$, we would have obtained
the same set of shifted partial products. This shows that simply
knowing the collections of shifted partial products does not permit
uniquely determining the multiplier and multiplicand. Recall that
this dilemma was alluded to in Section~\ref{sec:intro}, when discussing
pattern-matching based re-writing.

\end{example}



\subsection{Wallace tree multiplier}
%
Wallace tree\cite{wallace} decomposes the multiplication all the way down to single bits.
%
Let us consider bit-vectors $x$ and $y$ that are accessed in  blocks of $1$
bit and are of size $k$.
%
In a Wallace tree, a partial product is the multiplication of single
bits $x_i*y_j$.
%
The multiplication of single bits is the conjunction of the bits, i.e.,
$x_i \land y_j$.
%
There is no carry generated due to the multiplication of single bits.
%
The partial product $x_i*y_j$ is aligned with the $(i+j-2)$th bit of output.
%
Let us consider $o$th output bit.
%
All the partial products that are aligned to $o$ are summed using full adder 
and half adders.
%
The full adders are used
if more than two bits are available that are yet to be summed
and half adders are used if there are only two bits that are left to be summed.
%
The carry bits that are generated by the adders are aligned to
$(o+1)$th output bit.
%
The carry bits are summed to the partial products for $(o+1)th$ bit
using adders as illustrated in the following figure.

\begin{center}
  \begin{tikzpicture}[node distance=4cm,thick]

    \node[draw,rectangle, minimum width=2cm,minimum height=1cm] (a) {Adders};
    \node[draw,rectangle, minimum width=2cm,minimum height=1cm, right of=a] (b) {Adders};

    \draw[->] (a.south) -- node[right=1pt] {$o+1$} +(0,-.5);
    \draw[->] (b.south) -- node[right=1pt] {$o$} +(0,-.5);

    \draw[vecArrow] (b.220) |- ++(-1.5cm,-0.5) --node[right=1pt,yshift=-1cm,rotate = 90] {carry bits} ++(0,2.3cm) -| (a.40);

    \draw[vecArrow] ($ (a.140) + (0,0.6cm) $) --node[above=1pt,yshift = 1mm] {
      \begin{tabular}{c}
        partial\\
        products
      \end{tabular}
      } (a.140);
    \draw[vecArrow] ($ (b.140) + (0,0.6cm) $) --node[above=1pt,yshift = 1mm] {
      \begin{tabular}{c}
        partial\\
        products
      \end{tabular}
      } (b.140);

    \draw[vecArrow,gray] ($ (b.50) + (1cm,0.8cm) $) -| (b.50);
  \end{tikzpicture}  
\end{center}

The steps of Wallace tree multiplier can be summed up as:

\begin{itemize}

\item \textsc{Partial products generation} - Compute $n^2$ partial products, multiplying each bit of one of the arguments, by each bit of the other. Depending on position of the multiplied bits, the wires carry different weights, for example wire of bit carrying result of $a_{2}b_{3}$ has weight 32.
\item \textsc{Reduction} - Reduce the number of partial products of each weight to atmost two by layers of full and half adders.
\item \textsc{Final addition} - Group the wires in two numbers, and add them with a conventional adder.

\end{itemize}

The \textsc{reduction} step works as follows: As long as there are three or more wires with the same weight, add a following layer:

\begin{itemize}

\item Take any three wires with the same weights and input them into a full adder. The result will be an output wire of the same weight i.e $\sumOfFullAdder$ and an output wire with a higher weight i.e $\carryOfFullAdder$ for each three input wires.
\item If there are two wires of the same weight left, input them into a half adder. The result will be an output wire of the same weight i.e $\sumOfHalfAdder$ and an output wire with a higher weight i.e $\carryOfHalfAdder$.
\item If there is just one wire left, connect it to the next layer.
\end{itemize}


\begin{example}

Let us multiply two 4 bit bit-vectors $a_3a_2a_1a_0$ and $b_3b_2b_1b_0$ using the above steps:

\begin{enumerate}
\item Partial products generation: First we multiply every bit by every bit:
\begin{itemize}
\item weight 1 - $a_0b_0$
\item weight 2 - $a_0b_1$, $a_1b_0$
\item weight 4 - $a_0b_2$, $a_1b_1$, $a_2b_0$
\item weight 8 - $a_0b_3$, $a_1b_2$, $a_2b_1$, $a_3b_0$
\item weight 16 - $a_1b_3$, $a_2b_2$, $a_3b_1$
\item weight 32 - $a_2b_3$, $a_3b_2$
\item weight 64 - $a_3b_3$
\end{itemize}

\item Reduction layer 1:
\begin{itemize}
\item Pass the only weight-1 wire through, output: 1 weight-1 wire
\item Add a half adder for weight 2, outputs: 1 weight-2 wire, 1 weight-4 wire
\item Add a full adder for weight 4, outputs: 1 weight-4 wire, 1 weight-8 wire
\item Add a full adder for weight 8, and pass the remaining wire through, outputs: 2 weight-8 wires, 1 weight-16 wire
\item Add a full adder for weight 16, outputs: 1 weight-16 wire, 1 weight-32 wire
\item Add a half adder for weight 32, outputs: 1 weight-32 wire, 1 weight-64 wire
\item Pass the only weight-64 wire through, output: 1 weight-64 wire
\end{itemize}

\item Reduction layer 2: Add a full adder for weight 8, and half adders for weights 4, 16, 32, 64
\item Final addition - Group the wires into a pair of integers and an adder to add them.
\end{enumerate}
\end{example}

\begin{figure}[h]
\usetikzlibrary[arrows]
\centering
\begin{tikzpicture}[circuit logic US, minimum height=10mm]
\matrix[column sep = 7mm, row sep = 0.5mm]
{
\node (a3) {$a_3$}; &[-5mm] \node (a2) {$a_2$}; &[-5mm] \node (a1) {$a_1$}; &[-5mm] \node (a0) {$a_0$}; &[-5mm] \node (b3) {$b_3$}; &[-5mm] \node (b2) {$b_2$}; &[-5mm] \node (b1) {$b_1$}; &[-5mm] \node (b0) {$b_0$}; & \node (name1) {$PP$}; & \node (name2) {$RL1$}; & \node (name3) {$RL2$}; & & \node (name4) {$Add$};\\
&[-5mm] &[-5mm] &[-5mm] &[-5mm] &[-5mm] &[-5mm] &[-5mm] &[-5mm] \node[and gate] (and00) {$a_0b_0$}; & \\
&[-5mm] &[-5mm] &[-5mm] &[-5mm] &[-5mm] &[-5mm] &[-5mm] &[-5mm] \node[and gate] (and01) {$a_0b_1$}; & \node[draw,rectangle, minimum width=1.5cm,minimum height=1.5cm] (ha1) {HA1:wt2}; & \\
&[-5mm] &[-5mm] &[-5mm] &[-5mm] &[-5mm] &[-5mm] &[-5mm] &[-5mm] \node[and gate] (and10) {$a_1b_0$}; & & \node[draw,rectangle, minimum width=1.5cm,minimum height=1.5cm] (ha3) {HA3:wt4}; &\\
&[-5mm] &[-5mm] &[-5mm] &[-5mm] &[-5mm] &[-5mm] &[-5mm] &[-5mm] \node[and gate] (and02) {$a_0b_2$}; & \\
&[-5mm] &[-5mm] &[-5mm] &[-5mm] &[-5mm] &[-5mm] &[-5mm] &[-5mm] \node[and gate] (and11) {$a_1b_1$}; & \node[draw,rectangle, minimum width=1.5cm,minimum height=1.5cm] (fa1) {FA1:wt4}; &\\
&[-5mm] &[-5mm] &[-5mm] &[-5mm] &[-5mm] &[-5mm] &[-5mm] &[-5mm] \node[and gate] (and20) {$a_2b_0$}; & \\
&[-5mm] &[-5mm] &[-5mm] &[-5mm] &[-5mm] &[-5mm] &[-5mm] &[-5mm] \node[and gate] (and03) {$a_0b_3$}; & \\
&[-5mm] &[-5mm] &[-5mm] &[-5mm] &[-5mm] &[-5mm] &[-5mm] &[-5mm] \node[and gate] (and12) {$a_1b_2$}; & \node[draw,rectangle, minimum width=1.5cm,minimum height=1.5cm] (fa2) {FA2:wt8}; & \node[draw,rectangle, minimum width=1.5cm,minimum height=1.5cm] (fa4) {FA4:wt8}; &\\
&[-5mm] &[-5mm] &[-5mm] &[-5mm] &[-5mm] &[-5mm] &[-5mm] &[-5mm] \node[and gate] (and21) {$a_2b_1$}; & & & & \node[draw,rectangle, minimum width=1.5cm,minimum height=1.5cm] (adder) {Final Adder};\\
& & & & & & & & \node[and gate] (and30) {$a_3b_0$}; & & \node[draw,rectangle, minimum width=1.5cm,minimum height=1.5cm] (ha4) {HA4:wt16}; &\\
&[-5mm] &[-5mm] &[-5mm] &[-5mm] &[-5mm] &[-5mm] &[-5mm] &[-5mm] \node[and gate] (and13) {$a_1b_3$}; & \\
&[-5mm] &[-5mm] &[-5mm] &[-5mm] &[-5mm] &[-5mm] &[-5mm] &[-5mm] \node[and gate] (and22) {$a_2b_2$}; & \node[draw,rectangle, minimum width=1.5cm,minimum height=1.5cm] (fa3) {FA3:wt16}; &\\
&[-5mm] &[-5mm] &[-5mm] &[-5mm] &[-5mm] &[-5mm] &[-5mm] &[-5mm] \node[and gate] (and31) {$a_3b_1$}; & & \node[draw,rectangle, minimum width=1.5cm,minimum height=1.5cm] (ha5) {HA5:wt32}; &\\
&[-5mm] &[-5mm] &[-5mm] &[-5mm] &[-5mm] &[-5mm] &[-5mm] &[-5mm] \node[and gate] (and23) {$a_2b_3$}; & \\
&[-5mm] &[-5mm] &[-5mm] &[-5mm] &[-5mm] &[-5mm] &[-5mm] &[-5mm] \node[and gate] (and32) {$a_3b_2$}; & \node[draw,rectangle, minimum width=1.5cm,minimum height=1.5cm] (ha2) {HA2:wt32}; &\\
&[-5mm] &[-5mm] &[-5mm] &[-5mm] &[-5mm] &[-5mm] &[-5mm] &[-5mm] \node[and gate] (and33) {$a_3b_3$}; & & \node[draw,rectangle, minimum width=1.5cm,minimum height=1.5cm] (ha6) {HA6:wt64}; &\\
};

\draw (a3.south) -- ++(right:0mm) |- (and33.input 1);
\draw (a2.south) -- ++(right:0mm) |- (and23.input 1);
\draw (a1.south) -- ++(right:0mm) |- (and13.input 1);
\draw (a0.south) -- ++(right:0mm) |- (and03.input 1);

\draw (b3.south) -- ++(right:0mm) |- (and33.input 2);
\draw (b2.south) -- ++(right:0mm) |- (and32.input 1);
\draw (b1.south) -- ++(right:0mm) |- (and31.input 1);
\draw (b0.south) -- ++(right:0mm) |- (and30.input 1);

\draw (a0)  |- (and00.input 1);
\draw (b0)  |- (and00.input 2);

\draw (a0)  |- (and01.input 1);
\draw (b1)  |- (and01.input 2);

\draw (a1)  |- (and10.input 1);
\draw (b0)  |- (and10.input 2);

\draw (a0)  |- (and02.input 1);
\draw (b2)  |- (and02.input 2);

\draw (a1)  |- (and11.input 1);
\draw (b1)  |- (and11.input 2);

\draw (a2)  |- (and20.input 1);
\draw (b0)  |- (and20.input 2);

\draw (b3)  |- (and03.input 2);

\draw (a1)  |- (and12.input 1);
\draw (b2)  |- (and12.input 2);

\draw (a2)  |- (and21.input 1);
\draw (b1)  |- (and21.input 2);

\draw (a3)  |- (and30.input 2);

\draw (b3)  |- (and13.input 2);

\draw (a2)  |- (and22.input 1);
\draw (b2)  |- (and22.input 2);

\draw (a3)  |- (and31.input 2);

\draw (a3)  |- (and32.input 2);
\draw (b3)  |- (and23.input 2);

\draw (and01.output) |- (ha1);
\draw (and10.output) -- ++(right:2mm) |- (ha1.205) ;

\draw (and02.output) -- ++(right:1mm) |- (fa1.145);
\draw (and11.output)  |- (fa1);
\draw (and20.output) -- ++(right:1mm) |- (fa1.205);

\draw (and03.output) -- ++(right:1mm) |- (fa2.145);
\draw (and12.output)  |- (fa2);
\draw (and21.output) -- ++(right:1mm) |- (fa2.205);


\draw (and13.output) -- ++(right:1mm) |- (fa3.145);
\draw (and22.output)  |- (fa3);
\draw (and31.output) -- ++(right:1mm) |- (fa3.205);


\draw (and32.output) |- (ha2);
\draw (and23.output) -- ++(right:2mm) |- (ha2.155) ;

\draw[transform canvas={yshift=-5pt}] (ha1.east) -- ++(right:3mm) |- (ha3.155);

\draw[transform canvas={yshift=+5pt}] (fa1.east) -- ++(right:3mm) |- (ha3.205);

\draw[transform canvas={yshift=-5pt}] (fa1.east) -- ++(right:3mm) |- (fa4.145);

\draw[transform canvas={yshift=+5pt}] (fa2.east) |- (fa4.175);

\draw (and30.output) -- ++(right:25mm) |- (fa4.205);
\draw[transform canvas={yshift=-5pt}] (fa2.east) -- ++(right:4mm)|- (ha4.155);
\draw[transform canvas={yshift=+5pt}] (fa3.east) -- ++(right:3mm) |- (ha4.205);

\draw[transform canvas={yshift=-5pt}] (fa3.east) -- ++(right:3mm) |- (ha5.155);

\draw[transform canvas={yshift=+5pt}] (ha2.east) -- ++(right:3mm) |- (ha5.205);

\draw[transform canvas={yshift=-5pt}] (ha2.east) -- ++(right:3mm) |- (ha6.155);

\draw (and33.output) |- (ha6);

\draw (and00.output) -- ++(right:60mm) |- (adder.150);

\draw[transform canvas={yshift=+5pt}] (ha1.east) -- ++(right:35mm) |- (adder.160);

\draw[transform canvas={yshift=+5pt}] (ha3.east) -- ++(right:9mm) |- (adder.165);

\draw[transform canvas={yshift=-5pt}] (ha3.east) -- ++(right:8mm) |- (adder.165);

\draw[transform canvas={yshift=+5pt}] (fa4.east) -- ++(right:7mm) |- (adder.170);

\draw[transform canvas={yshift=-5pt}] (fa4.east) -- ++(right:6mm) |- (adder.175);

\draw[transform canvas={yshift=+5pt}] (ha5.east) -- ++(right:8mm) |- (adder.200);

\draw[transform canvas={yshift=-5pt}] (ha5.east) -- ++(right:9mm) |- (adder.205);

\draw[transform canvas={yshift=+5pt}] (ha4.east) -- ++(right:6mm) |- (adder.180);

\draw[transform canvas={yshift=-5pt}] (ha4.east) -- ++(right:7mm) |- (adder.185);

\draw[transform canvas={yshift=+5pt}] (ha6.east) -- ++(right:10mm) |- (adder.190);

\draw[transform canvas={yshift=-5pt}] (ha6.east) -- ++(right:11mm) |- (adder.195);


\end{tikzpicture}
\caption{Example 4}
\end{figure}


The benefit of the Wallace tree is that there are only $\mathcal{O}(\log n)$ reduction layers, and each layer has $\mathcal{O}(1)$ propagation delay. As making the partial products is $\mathcal{O}(1)$ and the final addition is $\mathcal{O}(\log n)$, the multiplication is $\mathcal{O}(\log n)$.





Both the above multiplication methods do not
fully specify the design.
%
Therefore, there are several ways to implement the multiplications.
%
Therefore, it is not trivial to verify that a hardware design indeed
implements a multiplication.

% \ashu{Needs more detailed discussion with example about how
% verification problem may contain such multipliers expanded out!}

%--------------------- DO NOT ERASE BELOW THIS LINE --------------------------

%%% Local Variables: 
%%% mode: latex
%%% TeX-master: "main"
%%% End: 

