We need to prove that each $[x*y = t]$ added in $\textsc{OurSolver}$
is a tautology.
%
First we will prove the correctness of $\textsc{getMultOperands}$.
%
If either of $x$ or $y$ is zero, we assume term $x*y$ is also simplified to zero.
\begin{thm}
 If $ x*y \in \textsc{getMultOperands}(\Lambda,w)$, then
 $$
 \Lambda_i = \{ x_1*y_{i-1},....,x_{i-1}*y_1 \}
 $$
  where $x_k$ and $y_k$ are the $k$th block of $x$ and $y$ of size
$w$, respectively.
\end{thm}
\begin{proof}
  After each iteration of the loop at line 8,
  if no backtracking triggered, the loop body ensures
  that the following holds at the end, which readers may easily check.
  \begin{equation}
    \label{eq:container-inv}
  \Lambda_i = \{ x_h*y_{i}, x_{h-1}*y_{i+1},....,x_{i}*y_h \}    
  \end{equation}
  Due to the above equation,
  if $x_j*y_k \in \Lambda_i$, $ i = h-j-k$.
  %
  If the program enters at line 25, it has a successful match and $i=1$.
  Since $h-l_x -l_y \geq 1$, $\Lambda_l = \{x_{l_x}*y_{l_x}\}$
  and $l = h - l_x - l_y$.
  %
  We choose $o \leq l$, and shift $x$ and $y$ according to lines 26-27.
  After the shift, we need to write equation~\eqref{eq:container-inv}
  as follows.
  \begin{equation}
    \label{eq:container-shifted}
    \Lambda_i = \{ x_{h-(l_x-o)}*y_{i-(l_y-l+o)},...,x_{i-(l_x-o)}*y_{h-(l_y-l+o)} \}.
  \end{equation}
  We can easily verify that the sum of the indexes in each of
  the partial products is $i$.
  %
  Since all $x_k$ is zero for $k > h-(l_x-o)$ and all $y_k$ is zero
  for $k > h-(l_y-l+o)$,
  we may rewrite equation~\eqref{eq:container-shifted}
 as follows.
  $$
  \Lambda_i = \{ x_1*y_{i-1},....,x_{i-1}*y_1 \}.
  $$
\end{proof}

\begin{thm}
  If $m*n\in$ \textsc{MatchLong}$(t)$, $[m*n = t]$ is a tautology.
\end{thm}
\begin{proof}
  We collect partial products with appropriate offsets $o$ at line 5.
  %
  The pattern of $t$ indicates that the net result is the sum of the 
  partial products with the respective offsets. 
  %
  $\textsc{getMultOperands}(\Lambda,w)$ returns the
  matches that produces the sums.
  %
  Therefore, $[m*n = t]$ is a tautology.
\end{proof}

\begin{thm}
  If $m*n\in$ \textsc{MatchWallaceTree}$(t)$, $[m*n = t]$ is a tautology.
\end{thm}
\begin{proof}
  %
  All we need to show that $t$ is summing the partial products stored in
  $\Lambda$.
  %
  The rest of the proof follows the previous theorem.

  Each bit $t_i$ must be the sum of the partial products $\Lambda_i$ and 
  the carry bits produced by the sum for $t_{i-1}$.
  %
  The algorithm identifies the terms that are added to obtain $t_i$
  and collects the intermediate results of the sum in $\Delta_{i}$.
  %
  We only need to prove that the terms that are not identified as
  partial products are carry bits of the sum for $t_{i-1}$.
  %
  Let us consider such a term $t$.
  %
  Let us suppose the algorithm identifies $t$ as an output of the carry
  bit circuit of a full adder (half adder case is similar) with inputs
  $a$, $b$, and $c$.
  %
  The algorithm also checks that $a$, $b$, $c$, $a \lxor b$ and
  $a \lxor b \lxor c$ are the intermediate
  results of the sum for $t_{i-1}$.
  %
  Therefore, $t$ is one of the carry bits.
  %
  Since $a$, $b$, $a \lxor b$ and $c$ are removed from $\Delta_{i-1}$
  after the match of the adder,
  all the identified adders are disjoint.
  %
  Since we require that all the elements of $\Delta_{i-1}$  are eventually
  removed except $t_{i-1}$, all carry bits are added to obtain $t_i$.
  %
  Therefore, $\Lambda$ has the expected partial products of a Wallace tree.
\end{proof}

%--------------------- DO NOT ERASE BELOW THIS LINE --------------------------

%%% Local Variables:
%%% mode: latex
%%% TeX-master: "main"
%%% End:
