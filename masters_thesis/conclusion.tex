We have shown how adding tautological assertions that assert the
equivalence of different representations of bit-vector multiplication
can siginificantly improve the performance of SMT solvers.  We are
currently extending our procedure to support Booth multiplier and
other more complex arithmetic patterns.  We are also working to add
proof generation support for the added tautological assertions.  We
could not include proof generation in this work, since the basic
infrastructure of proof generation is missing in $\zthree$~bit-vector
rewriter module.
%
%We are also planning to request \zthree~team to adopt our modification in their
%standard distribution.

We plan to extend our work to problems involving different
combinations of multiplications. For example, $(X*(Y*Z))*W$ involves
three multiplications. We could chose to specify the implementation of
the three multiplications to be either of Long, Wallace, Booth
multiplication or an unspecified multiplication. To enable the pattern
detection in arbitrary arithmetic formulas, we are working to modify
the word level reasoning in the solvers.
We could further extend our work to complicated word level formulas involving bit vector multiplication like $X*X + Y*Y$, $X*(Y+Z)$, $(X*(Y*Z))*W$.

One benchmark which shows no improvements within time limits is mult\_3\_operands. The benchmark timed out in all the cases and hence is not reported. While all the other benchmarks above check for equality of multiplication of two bit vectors, mult\_3\_operands is an attempt to see if the approach works for multiplication of 3 operands which is carried out as 2 long multiplications. Our observations indicate that none of the solvers are able to infer the structural similarity based on the 2 tautologies added for the 2 long multiplications.
None of the solvers were able to make use of the added tautologies(one for each long multiplication). 
The problem seems to be on two fronts: one is that to check equality of two terms, the solvers want exact structural similarity and secondly the infrastructure of using one assertion to infer the satisfiability of another does not seem to be well developed. 
We aim to bridge this gap as part of our future work.



%--------------------- DO NOT ERASE BELOW THIS LINE --------------------------

%%% Local Variables: 
%%% mode: latex
%%% TeX-master: "main"
%%% End: 
