% Bit-vector is important
%
In recent years, SMT (Satisfiability Modulo Theories) solving has
emerged as a key technology for testing, analysis and verification of
hardware and software systems.  An SMT solver combines the reasoning
power of multiple first-order theory solvers to check the
satisfiability of a formula in a combination of theories.
%% Examples of theories commonly supported in modern SMT solvers are
%% those of fixed-width bit-vectors, arrays, lists, strings and the like.
%% Since reasoning in these individual theories is often much more
%% efficient than reasoning about the bit-level representation of the
%% corresponding data types, techniques based on SMT solving hold a lot
%% of promise as far as scaling to large applications is concerned.  The
%% impressive progress made in SMT solving over the last two
%% decades~\cite{smtprogress} has also substantially lived up to this
%% hope. 
An important theory, formulas of which occur naturally when reasoning
about systems with finite-precision data, is the theory of fixed-width
bit-vectors (or words).  Indeed, several important technologies,
viz. bounded model checking of RTL
designs~\cite{hwcbmc,boolector,ebmc} and of programs with
bounded-width integers~\cite{cbmc,corral,boogie}, symbolic trajectory
evaluation of RTL designs~\cite{wste}, constrained test generation for
programs and RTL designs~\cite{crv,concolic,dart} etc. rely crucially
on checking the satisfiability of a formula in a combination of
theories that includes the theory of bit-vectors.  Improving the
efficiency of bit-vector solvers can therefore help improve the
performance of several technologies that use SMT solvers.

If the system under analysis involves word-level arithmetic
operations, viz.  multiplying the contents of two registers or adding
two bounded-width integers, it is desirable to reason about the
word-level arithmetic operations directly, i.e. without decomposing
them into sub-operations.  This allows the bit-vector theory solver to
use simplification rules that encode properties of word-level
arithmetic operators, viz. $x + y = y + x$ or $x * 0 = 0$, when
checking the satisfiability of a formula.  Unfortunately, word-level
arithmetic operators may not be implemented as monolithic operators,
especially if the system under analysis is a hardware system.
Depending on the complexity of the operation, the operator may be
decomposed and implemented in one of several ways, guided by reasons
related to performance, layout, component count etc.  For example,
addition of two words may be implemented using a ripple-carry adder,
carry-propagate adder, carry-save adder, etc.  Similarly, one can use
a grade-school multiplier (implementing the standard long
multiplication), Booth-encoded multiplier, Wallace-tree multiplier
etc. to implement multiplication of two words.  While bit-vector
solvers in state-of-the-art SMT
solvers~\cite{zthree,cvcfour,boolector,mathsat} can reason about
explicitly specified word-level arithmetic operators efficiently,
empirical evidence~\cite{wste}) shows that their performance degrades
significantly when the operator is not specified explicitly, but as a
composition of several sub-operators (as in a carry-save adder, or a
grade-school multiplier).  This motivates us to ask \emph{if we can
  preprocess an SMT formula efficiently and provide ``hints'' to the
  SMT solver to help check satisfiability efficiently.}  

At first sight, the above problem appears to be one of
reverse-engineering a decomposed implementation (encoded in the SMT
formula obtained from analyzing a system) of a complex word-level
operation, which can be achieved by matching a sub-formula with a
pre-specified ``pattern''.  A closer inspection, however, reveals that
there are several subtle complications. First, and contrary to
intuition, the same decomposed implementation may correspond to
multiple reverse-engineered word-level operations.  A simple example
of this is obtained by considering the high-school multiplication
algorithm.  Second, even if we are able to recover a complex
word-level operation from a sub-formula corresponding to a
decomposition in the given SMT formula, rewriting the sub-formula with
the recovered operation may not correlate with an improved
satisfiability check for the overall SMT formula.  This can happen for
two reasons:
\begin{itemize}
\item When multiple word-level operations can be recovered from the
  same sub-formula, one cannot decide a priori which word-level
  operation will be beneficial for the overall satisfiability check.
\item Two sub-formulas that share further sub-formulas may match two
  pre-specified patterns, and thereby yield two different word-level
  operations.  Rewriting one of the sub-formulas with its
  corresponding word-level operation removes the shared sub-formulas,
  and thereby precludes rewriting the other sub-formula with its
  corresponding word-level operation.  In general, it may not be
  possible to determine a priori whether one of these re-writes helps
  the overall satisfiability checking more than the other.  In fact,
  re-writing both of the sub-formulas may be desirable, but is 
  precluded if we re-write one of the sub-formulas.
\item The recovery is a local ``peep-hole'' operation that looks only
  at the sub-formula matching a specific ``pattern''.  Specifically,
  the recovery is oblivious of the context in which the sub-formula
  appears in the overal SMT formula, and rewriting the sub-formula
  with the recovered word-level operation may not help (and may even
  hurt) the overall satisfiability check.  
\item In general, the sub-formula matching a pre-specified ``pattern''
  may help in simplifying some part of the SMT formula, while the
  recovered word-level operation may help in simplifying some other
  part of the SMT formula.  Re-writing the matched sub-formula with
  the recovered operation therefore precludes using the sub-formula
  to simplify the SMT formula.
\end{itemize}


%% Many hardware and software verification problems are translated to the
%% satisfiability of quantifier free bit-vector(QF\_BV)
%% formulas~\cite{hardware,cbmc,more}.
%
There are several SMT solvers that implement satisfiability of 
QF\_BV formulas, e. g., $\zthree$~\cite{z3}, $\cvcfour$~\cite{cvc4}, 
and many others.
%
A typical SMT solver proceeds by applying several simplification rules
on the input formula.
%
If the solver can not simplify a formula anymore and then it
translates the formula bit-blasting.

% Multiplication is widely used
%
In several applications, such as control and microprocessor designs,
the multiplications of bit-vectors are widely used.
%
The bit-vector multiplications are particularly hard for SMT solvers.
%

% A multiplication may be implemented in pieces 
%
Since multiplication is an expensive operation, it is often broken down
in small pieces and the results are added with appropriates offsets.
%
There are several algorithms for breaking down multiplications, for
example, high-school grade multiplier, \ashu{other multipliers}, Booth
multiplier, etc.
%
If a multiplication is implemented using one of the algorithm then the
solver may not be able to recognize that the net effect of the
algorithm is the multiplication of some bit-vectors.
%
 



% SMT solvers do not recognize the pieces and the assembly of pieces


% We have a pattern for recognizing multipliers


% Two ways to implement : add as eager theory axioms or rewriting of the inputs


% Implemented in Z3 and benchmark

% Results


%--------------------- DO NOT ERASE BELOW THIS LINE --------------------------

%%% Local Variables: 
%%% mode: latex
%%% TeX-master: "main"
%%% End: 
