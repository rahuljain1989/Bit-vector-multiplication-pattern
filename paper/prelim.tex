In this section, we will present syntax of
the theory of quantifier-free bit vector formulas(\qfbv),
basics of solving methods used by SMT solvers,
and the multiplication procedures that are of our interest. 



\subsection{\qfbv~syntax}

Syntax of bit-vector formulas

\subsection{Multipliers}

\subsubsection{Grade school multiplication}

Describe it in our $\qfbv$

\ashu{2-3 line why this definition and what to expect in the words already 
introduced}

\rahul{ We need to do this}

\begin{df}
  
\end{df}

Some discussion about the definition.

\begin{example}
  
\end{example}


\subsubsection{Wallace tree multiplier}

\subsubsection{Booth multiplier}


%\subsection{Verification problem to SMT formula}



%--------------------- DO NOT ERASE BELOW THIS LINE --------------------------

%%% Local Variables: 
%%% mode: latex
%%% TeX-master: "main"
%%% End: 
