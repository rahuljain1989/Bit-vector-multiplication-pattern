In this section, we will present syntax of
the theory of quantifier-free bit vector formulas(\qfbv),
basics of solving methods used by SMT solvers,
and the multiplication methods that are of our interest. 

\subsection{\qfbv~syntax}

A bit-vector is a fixed sequence of bits.
%
We will denote bit vectors by $x$,$y$,$z$...
%
We need to refer to the blocks of bits in 
bit-vectors.
Therefore, we may declare that a bit-vector $x$
is accessed in blocks of size $w$.
%
We write $x_i$ to denotes $i$th block from
least significant bit (LSB).


A $\qfbv$~term $t$ and formula $F$ may be constructed using
the following grammar.
\begin{align*}
t ::= & t * t \mid t + t \mid x \mid n^w \mid t \concat t \mid ....\\
F ::= & t = t \mid t \bowtie t | \lnot F | F \lor F | F \land F | F \lxor F \mid ... 
\end{align*}
where $x$ is a bit-vector variable, 
$n^w$ is a constant number represented by $w$ bits,
$\bowtie \in \{\leq , <, \geq, > \}$, and
$\concat$ is a binary operator that concatenates bit-vectors.
%
Here we are presenting only the  part of the theory
of $\qfbv$ that is relevant to our discussion.
%
In this paper, all the arithmetic operators are unsigned.
%
All the inputs and outputs of arithmetic bit-vector operators 
have same bit length.
%
Let $len(t)$ denote the bit length of a term $t$.

\subsection{SMT solvers for \qfbv}

SMT(satisfiability modulo theory)
solvers are the specialized solvers that solve 
formulas of a given theory.
%
The leading SMT solvers for \qfbv~apply several simplification
passes followed by bit-blast, i.e. translating input to Boolean
satisfiability problem.
%
The bit-blasted SAT problem is solved using conflict driven clause
learning(CDCL) based procedures.
%
Some of the SMT solvers that implement \qfbv~are
\zthree, \boolector, \cvcfour, etc.

For our algorithm, we assume that a  $\qfbv$~SMT
solver $\textsc{SMTSolver}$ 
with the standard interface is available.
%
The interface includes a function $add(F)$ that adds a formula
into the solver and a function $checkSat()$ that checks the
satisfiability of the formulas added so far. 

\subsection{Multipliers}

Multiplication is an expensive operation to implement in hardware.
%
There are several designs of multipliers for varying
resource constraints.
%
If one can have a large number of gates then Wallace tree
multiplier can be used.
%
Otherwise, one may break down the multiplication task in
small multiplications and combine the results appropriately.
%
For example, grade school multiplier and Booth multiplier.
%
Here, we will discuss Wallace tree and Grade school multiplier.

\subsubsection{Grade school multiplication}

Let us consider bit-vectors $x$ and $y$ that are accessed in the blocks of $w$
bits and are of size $kw$.
%
The grade school multiplier breaks down the multiplication $kw$ bits
into chunks of $w$ bits multiplication, called {\em partial products}.
%
The partial products are summed with appropriate offsets to obtain
the final result.
%
The following notation is typically used to illustrate
the grade school multiplication.
%
\begin{center}
\begin{tabular}{c@{\quad}c@{\quad}c@{\quad}c@{\quad}c@{\quad}c@{\quad}c}
  &&& $x_{k}$ & ... & $x_1$&\\ 
  &&& $y_{k}$ & ... & $y_1$&$*$\\ \hline
  &&&$x_k*y_1$& ... & $x_1*y_1$&\\
  &&$\iddots$&$\vdots$& $\iddots$ && \\
  &$x_k*y_k$& ... &$x_1*y_k$&  & +&\\\hline
\end{tabular}  
\end{center}
$x_i * y_j$ are the partial products.
%
$x_i*y_j$ is left shifted $(i+j-2)*w$ bits. 
%
In the above scheme all the partial products that have same offset are 
aligned in single column.
%
After the shifts, all the partial results are added in some order.
%
Let us clarify that the needed bit-width of the partial products should be $2w$.
%
Formally, $x_i * y_j$ denotes $(0^w \concat x_i)*(0^w \concat y_j)$.
%
Therefore, the bits of the partial products in neighbouring columns overlap
and they can not be simply concatenated.
%
The scheme does not specify the order of the addition
of the shifted partial products.
%
Therefore, there are several design possible for a given $k$ and $w$.

\begin{example}
  Consider bit-vectors $v_1,v_2,u_1$, and $u_2$ of length 2.
  Let us apply grade school multiplication in multiplying
  $v_2 \concat 0^2 \concat v_1$ and $u_2 \concat v_2 \concat u_1$.
  We obtain the following partial products.
\begin{center}
\begin{tabular}{c@{\quad}c@{\quad}c@{\quad}c@{\quad}c@{\quad}c@{\quad}c}
  &&& $v_2$ & $0^2$ & $v_1$&\\ 
  &&& $u_2$ & $v_2$ & $u_1$&$*$\\ \hline
  &&&$v_2*u_1$& $0^4$ & $v_1*u_1$&\\
  &&$v_2*v_2$&$0^4$& $v_1*v_2$ && \\
  &$v_2*u_2$& $0^4$ &$v_1*u_2$&  & +&\\\hline
\end{tabular}
\end{center}
We need to sum the partial products. However, if their non-zero bits 
do not overlap then we can simply concatenate them.
%
And finally we may sum the concatenated vectors.
%
The following is one the combination of the concatenation and 
summation of the above multiplication.
$$
( 0^4 \concat v_1*u_2 \concat v_1*u_1)+ (v_2*u_2 \concat v_2*u_1 \concat 0^4)+ (0^2 \concat v_2*v_2 \concat v_1*v_2 \concat 0^2)
$$
\end{example}


\subsubsection{Wallace tree multiplier}
%
Wallace tree breaks down the multiplication all the way to single bits.
%
Let us consider bit-vectors $x$ and $y$ that are accessed in the blocks of $1$
bits and are of size $k$.
%
In a Wallace tree, the partial products are multiplication of single
bits $x_iy_j$.
%
The multiplication of single bits is the conjunction of the bits, i.e.,
$x_i \land y_i$.\footnote{1-bit bit vectors may be viewed as boolean values therefore we may apply logical and.}
%
There is no carry generated due to the multiplication of the single bits.
%
We need to align the partial product $x_iy_j$ to $(i+j-2)$th bit of output.
%
Consider $o$th output bit.
%
All the partial products that are align to $o$ are summed using full adder 
and half adders.
%
The full adders are used
if more than three bits are available that are yet to be added
and half adder are used if there are only two bits that are left to be added.
%
The adders generate carry bits that are aligned to $(o+1)$th bit,
which are added to the partial products for $(o+1)th$ bit using the 
adders as illustrated in the following figure.

\begin{center}
  \begin{tikzpicture}[node distance=4cm,thick]

    \node[draw,rectangle, minimum width=2cm,minimum height=1cm] (a) {Adders};
    \node[draw,rectangle, minimum width=2cm,minimum height=1cm, right of=a] (b) {Adders};

    % \draw (0,0)  rectangle +(2,1);
    % \draw (3,0)  rectangle +(2,1);

    \draw[->] (a.south) -- node[right=1pt] {$o+1$} +(0,-.5);
    \draw[->] (b.south) -- node[right=1pt] {$o$} +(0,-.5);


    \draw[vecArrow] (b.220) |- ++(-1.5cm,-0.5) --node[right=1pt,yshift=-1cm,rotate = 90] {carry bits} ++(0,2.3cm) -| (a.40);

    \draw[vecArrow] ($ (a.140) + (0,0.6cm) $) --node[above=1pt,yshift = 1mm] {
      \begin{tabular}{c}
        partial\\
        products
      \end{tabular}
      } (a.140);
    \draw[vecArrow] ($ (b.140) + (0,0.6cm) $) --node[above=1pt,yshift = 1mm] {
      \begin{tabular}{c}
        partial\\
        products
      \end{tabular}
      } (b.140);

    \draw[vecArrow,gray] ($ (b.50) + (1cm,0.8cm) $) -| (b.50);


  % \draw[vecArrow] (k) |- (c);
  \end{tikzpicture}  
\end{center}


%\subsubsection{Booth multiplier}


\subsection{Verification problem to SMT formula}

We have observed that both the above multiplication method do not
fully specified the design.
%
Therefore, there are several ways to implement the multiplications.
%
For any hardware design containing multiplication,
one needs to verify that the design indeed implements the intended
multiplication.

\ashu{Needs more detailed discussion with example about how
verification problem may contain such multipliers expanded out!}

%--------------------- DO NOT ERASE BELOW THIS LINE --------------------------

%%% Local Variables: 
%%% mode: latex
%%% TeX-master: "main"
%%% End: 
