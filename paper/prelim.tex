In this section, we will present syntax of
the theory of quantifier-free bit vector formulas(\qfbv),
basics of solving methods used by SMT solvers,
and the multiplication methods that are of our interest. 

\subsection{\qfbv~syntax}

A bit-vector is a fixed sequence of bits.
%
We will denote bit vectors by $x$,$y$,$z$...
%
A $\qfbv$~term $t$ and formula $F$ may be constructed using
the following grammar.
\begin{align*}
t ::= & t * t \mid t + t \mid x \mid n \mid t \concat t \mid ....\\
F ::= & t = t \mid t \bowtie t | \lnot F | F \lor F | F \land F | F \lxor F \mid ... 
\end{align*}
where $x$ is a bit-vector variable, 
$n$ is a constant number represented in some length of bits,
$\bowtie \in \{\leq , <, \geq, > \}$, and
$\concat$ is a binary operator that concatenates bit-vectors.
%
Here we have presented only the  part of the theory
of $\qfbv$ that is relevant to our discussion.
%
All the inputs and outputs of arithmetic bit-vector operators 
have same bit length.
%
Let $len(t)$ denote the bit length of a term $t$.

\subsection{SMT solvers for \qfbv}

The leading SMT solvers for \qfbv~apply several simplification
passes followed by bit-blast, i.e. translating input to Boolean
satisfiability problem.
%
The bit-blasted SAT problem is solved using conflict driven clause
learning(CDCL) based procedures.
%
Some of the SMT solvers are \zthree.

We assume that an  $\qfbv$~SMT solver $\textsc{SMTSolver}$ 
with the standard interface is available.
%
The interface includes a function $add(F)$ that adds a formula
into the solver and a function $checkSat()$ that checks the
satisfiability of the formulas added so far. 

\subsection{Multipliers}



\subsubsection{Grade school multiplication}

Describe it in our $\qfbv$

\ashu{2-3 line why this definition and what to expect in the words already 
introduced}

\rahul{ We need to do this}

\begin{df}
  
\end{df}

Some discussion about the definition.

\begin{example}
  
\end{example}


\subsubsection{Wallace tree multiplier}

\subsubsection{Booth multiplier}


%\subsection{Verification problem to SMT formula}



%--------------------- DO NOT ERASE BELOW THIS LINE --------------------------

%%% Local Variables: 
%%% mode: latex
%%% TeX-master: "main"
%%% End: 
