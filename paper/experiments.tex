%
We have implemented\footnote{\url{https://github.com/rahuljain1989/Bit-vector-multiplication-pattern}} our algorithms as a part of the $\zthree$ SMT solver.
%
We evaluate the performance of our algorithms using benchmarks that are
derived from hardware verification problems.
%
We compare our tool with $\zthree$, $\boolector$ and $\cvcfour$.
%
Our experiments show that while the solvers time out on most of the
benchmarks without our heuristic, a portfolio solver using our heuristic
produces results within the prescribed time limit.

\paragraph{\bf Implementation}
We have added about 1500 lines of code in the bit vector rewrite module of $\zthree$ because it allows easy access to the abstract syntax tree of the input formula. We call this version of $\zthree$ as instrumented-$\zthree$.
%
An important aspect of the implementation is its ability to exit early if the match is going to fail.
%
We implemented various preliminary checks including the ones mentioned in Algorithm 1. For example, we ensure that the size of $\Lambda_i$ is upper bounded appropriately as per the scheme of long multiplication. We exit as soon as the upper bound is violated. 
%
We have implemented three versions of $\ourtool$ by varying the choice of $\textsc{SMTSolver}$.
%
We used  $\zthree$, $\boolector$, and $\cvcfour$ for the variations. 
%

In each case we stop the instrumented-$\zthree$ solver after running our matching algorithms,
print the learned tautologies in a file along with the input formula, and
run the solvers in a separate process on the pre-processed formula. 
The time taken to run our matching algorithms and generate the pre-processed formula is less than one second across all our benchmarks, and hence is not reported. 
We also experimented by running instrumented-$\zthree$ standalone and found the run times to be similar to that of $\zthree$ running on the pre-processed formula; hence the run times for instrumented-$\zthree$ are not reported. We use the following versions of the solvers: $\zthree$(4.4.2), $\boolector$(2.2.0), $\cvcfour$(1.4) for our experiments.

\paragraph{\bf Benchmarks}
%
Our experiments were run on 20 benchmarks. 
%
Initially, we considered the motivating example described in Section~\ref{sec:intro} involving long multiplication that was not solved by any
of the solvers in 24 hours.
%
This example inspired our current work and to evaluate it we generated several similar benchmarks.
%
For long multiplication, we generated benchmarks by varying three characteristics. Firstly the total bit-width of the input bit-vectors, secondly the width of each block, and thirdly assigning specific blocks as equal or setting them to zero.
%
Our benchmarks were written in $\sysver$ and fed to STEWord~\cite{wste}, a hardware verification tool.
%
STEWord takes a $\sysver$ design and a specification of how it is to be driven
as input, and generates an SMT formula in SMT1 format. %the corresponding SMT1 formula.
%
We convert the SMT1 formula to SMT2 format using $\boolector$.
%
In the process, $\boolector$ extensively simplifies the input formula but retains the overall structure.
%
We have generated benchmarks also for Wallace tree multiplier similar to the long multiplication.
%
For $n$-bit Wallace tree multiplier, we have written a script that takes $n$ as input and generates all the files needed as input by STEWord. All our benchmarks correspond to the system implementation satisfying the specified property: in other words, the generated SMT formulas were unsatisfiable. For satisfiable formulas the solver was able to find satisfying assignments relatively quickly, both with and without our heuristic. Hence, we do not report results on satisfiable formulas.
%

\paragraph{\bf Results}
%
We compare our tool with $\zthree$, $\boolector$ and $\cvcfour$.
%
In Tables~\ref{tbl:time}-\ref{tbl:cd}, we present the results of the experiments.
%
We chose timeout to be 3600 seconds.
%
In Table~\ref{tbl:time}, we present the timings of the long multiplication and
Wallace tree multiplier experiments.
%
The first 13 rows correspond to
the long multiplication experiments.
%
The columns under $\textsc{SMTSolver}$ are the run times of the
solvers to prove the unsatisfiability of the input benchmark.
%
The solvers timed out on most of the benchmarks.
%

\begin{table}[t]
\centering
\caption{Multiplication experiments. Times are in seconds.
\textsc{Portfolio} columns is the least timing among the solvers previous .}
\label{my-label}
\begin{tabular}{|c|c|c|c|c|c|c|c|c|}
\hline
                      & \multicolumn{4}{c|}{\textsc{SMTSolver}}                    & \multicolumn{4}{c|}{$\ourtool$}                              \\ \hline
Benchmark             & {\scriptsize $\zthree$} & ${\scriptsize \boolector}$ & {\scriptsize $\cvcfour$} & {\scriptsize \textsc{Portfolio}} & {\scriptsize $\zthree$} & {\scriptsize $\boolector$} & {\scriptsize $\cvcfour$} & {\scriptsize \textsc{Portfolio}} \\ \hline
base                  & 184.3    & 42.2         & 16.54      & 16.54                & 0.53      & 43.5         & 0.01       & 0.01                 \\ \hline
ex1                   & 2.99      & 0.7          & 0.36       & 0.36                 & 0.33      & 0.8          & 0.01       & 0.01                 \\ \hline
ex1\_sc         & t/o       & t/o          & t/o        & t/o                  & 1.75      & t/o          & 0.01       & 0.01                 \\ \hline
ex2                   & 0.78      & 0.2          & 0.08       & 0.08                 & 0.44      & 0.3          & 0.01       & 0.01                 \\ \hline
ex2\_sc         & t/o       & 1718       & 2826    & 1718               & 3.15      & 1519       & 0.01       & 0.01                 \\ \hline
ex3                   & 1.38      & 0.3          & 0.08       & 0.08                 & 0.46      & 0.7          & 0.01       & 0.01                 \\ \hline
ex3\_sc         & t/o       & 1068       & t/o        & 1068               & 3.45      & 313.2        & 0.01       & 0.01                 \\ \hline
ex4         & 0.46      & 0.2          & 0.03       & 0.03                 & 0.82      & 0.2          & 0.01       & 0.01                 \\ \hline
ex4\_sc     & 287.3    & 62.8         & 42.36      & 42.36                & 303.6    & 12.8         & 0.01       & 0.01                 \\ \hline
sv\_assy              & t/o       & t/o          & t/o        & t/o                  & 0.07      & t/o          & 0.01       & 0.01                 \\ \hline
mot\_base   & t/o       & t/o          & t/o        & t/o                  & 13.03     & 1005       & 0.01       & 0.01                 \\ \hline
mot\_ex1 & t/o       & t/o          & t/o        & t/o                  & 1581   & 13.8         & 0.01       & 0.01                 \\ \hline
mot\_ex2 & t/o       & t/o          & t/o        & t/o                  & 2231   & 13.7         & 0.01       & 0.01                 \\ \hline \hline
wal\_4bit  & 0.09      & 0.05         & 0.02       & 0.02                 & 0.09      & 0.1          & 0.04       & 0.09                 \\ \hline
wal\_6bit  & 2.86      & 0.6          & 0.85       & 0.6                  & 0.28      & 0.8          & 14.36      & 0.28                 \\ \hline
wal\_8bit  & 209.8    & 54.6         & 225.1     & 54.6                 & 0.59      & 30.0         & 3471    & 0.59                 \\ \hline
wal\_10bit & t/o       & 1523       & t/o        & 1523               & 1.03      & 98.6         & t/o        & 1.03                 \\ \hline
wal\_12bit & t/o       & t/o          & t/o        & t/o                  & 1.55      & 182.3        & t/o        & 1.55                 \\ \hline
wal\_14bit & t/o       & t/o          & t/o        & t/o                  & 2.27      & 228.5        & t/o        & 2.27                 \\ \hline
wal\_16bit & t/o       & t/o          & t/o        & t/o                  & 2.95      & 481.7        & t/o        & 2.95                 \\ \hline

\end{tabular}
\end{table} 

%%% Local Variables:
%%% mode: latex
%%% TeX-master: "main"
%%% End:


The next three columns present the run times of the three versions of
$\ourtool$ to prove the satisfiability of the benchmarks.
%
$\ourtool$ with $\cvcfour$ makes best use of the added tautologies.
%
$\cvcfour$ is quickly able to infer that the input formula and the
added tautologies are negations of each other justifying the timings.
%
$\ourtool$ with $\boolector$ and $\zthree$ does not make the above
inference, leading to more running times.
%
$\boolector$ and $\zthree$ bit blast the benchmarks having not been
able to detect the structural similarity.
%
However, the added tautologies help $\boolector$ and $\zthree$ to
reduce the search space, after the SAT solver is invoked on the
bit-blasted formulas.


The last 7 rows correspond to the Wallace tree multiplier experiments.
%
Since the multiplier involves a series of half and full adders, the
size of the input formula increases rapidly as the bit vector width
increases.
%
Despite the blowup in the formula size, $\ourtool$ with $\zthree$ is
quickly able to infer that the input formula and the added tautology
are negations of each other.
%
However, $\ourtool$ with $\boolector$ and $\cvcfour$ do not make the
inference, leading to larger run times.
%
This is because of the syntactic structure of the learned tautology
from our implementation inside $\zthree$.
%
The input formula has `and' and `not' gates as its building blocks,
whereas $\zthree$ transforms all `ands' to `ors'.
%
Therefore, the added tautology has no `ands'.
%
The difference in the syntactic structure between the input formula
and the added tautology makes it difficult for $\boolector$ and
$\cvcfour$ to make the above inference.
%

We have seen that the solvers sometimes fail to apply word level
reasoning even after adding the tautologies.
%
In such cases, the solvers bit blast the formula and run a SAT solver.
%
In Table~\ref{tbl:cd}, we present the number of conflicts and decisions within
the SAT solvers.
%
The number of conflicts and decisions on running $\ourtool$ with the
three solvers, are considerably less than their $\textsc{SMTSolver}$
counterparts in most of the cases.
%
This demonstrates that the tautologies also help in reducing the
search inside the SAT solvers. 
%
$\ourtool$ with $\cvcfour$ has zero conflicts and
decisions for all the long multiplication experiments, because the 
word level reasoning solved the benchmarks.
%
Similarly, $\ourtool$ with $\zthree$ has zero conflicts and decisions
for all the Wallace tree multiplier experiments.

% \ashu{
% %
% We also applied $\ourtool$ with $\zthree$ on SMT-LIB benchmarks and compared 
% with the original $\zthree$.
% %
% $\ourtool$ was able to handle $..\%$ of the benchmarks.
% %
% We observed timings for $..\%$ of the benchmarks did not change more than $5\%$,
% $..\%$ of the benchmarks increased more than $5\%$,
% and 
% $..\%$ of the benchmarks decreased more than $5\%$.
% }

\begin{sidewaystable}[ph!]
\centering
\caption{Conflicts and decisions in the experiments. M stands for millions. k stands for thousands. }
\label{tbl:cd}
\begin{tabular}{|c|c|c|c|c|c|c|c|c|c|c|c|c|}
\hline
                      & \multicolumn{6}{c|}{\textsc{SMTSolver}}                                                            & \multicolumn{6}{c|}{$\ourtool$}                                                                      \\ \hline
Benchmark             & \multicolumn{2}{c|}{$\zthree$} & \multicolumn{2}{c|}{$\boolector$} & \multicolumn{2}{c|}{$\cvcfour$} & \multicolumn{2}{c|}{$\zthree$} & \multicolumn{2}{c|}{$\boolector$} & \multicolumn{2}{c|}{$\cvcfour$} \\ \hline
                      & Conflicts      & Decisions     & Conflicts       & Decisions       & Conflicts      & Decisions      & Conflicts      & Decisions     & Conflicts       & Decisions       & Conflicts      & Decisions      \\ \hline
base                  & 172k         & 203k        & 170k          & 228k          & 127k         & 148k         & 724            & 1433          & 148k          & 194k          & 0              & 0              \\ \hline
ex1                   & 7444           & 9065          & 7320            & 9892            & 8396           & 10k          & 474            & 890           & 7090            & 9558            & 0              & 0              \\ \hline
ex1\_sc         & t/o ---        & t/o ---       & t/o 5.6M     & t/o 7.7M     & t/o 2.1M    & t/o 2.3M    & 2564           & 5803          & t/o 5M     & t/o 6.8M     & 0              & 0              \\ \hline
ex2                   & 2067           & 2599          & 1789            & 2612            & 2360           & 3374           & 919            & 1420          & 1747            & 2526            & 0              & 0              \\ \hline
ex2\_sc         & t/o ---        & t/o ---       & 3.3M         & 4.9M         & 1.9M        & 2.3M        & 5076           & 8981          & 2.7M         & 4.3M         & 0              & 0              \\ \hline
ex3                   & 4109           & 5402          & 1682            & 3166            & 3374           & 4754           & 905            & 1321          & 3882            & 7305            & 0              & 0              \\ \hline
ex3\_sc        & t/o ---        & t/o ---       & 3.8M         & 5.9M         & t/o 2.9M    & t/o 3.6M    & 4814           & 9012          & 805k          & 1.4M         & 0              & 0              \\ \hline
ex4         & 647            & 801           & 612             & 715             & 463            & 588            & 630            & 918           & 405             & 519             & 0              & 0              \\ \hline
ex4\_sc      & 143k         & 165k        & 130k          & 165k          & 110k         & 130k         & 115k         & 138k        & 67k           & 114k          & 0              & 0              \\ \hline
sv\_assy              & t/o ---        & t/o ---       & t/o 5.3M     & t/o 9.8M     & t/o 1.8M    & t/o 2.5M    & 0              & 0             & t/o 4.7M     & t/o 9.4M     & 0              & 0              \\ \hline
mot\_base   & t/o ---        & t/o ---       & t/o 6M     & t/o 10M    & t/o 2.2M    & t/o 2.9M    & 12k          & 30k         & 2.4M         & 5.5M         & 0              & 0              \\ \hline
mot\_ex1 & t/o ---        & t/o ---       & t/o 4.4M     & t/o 6.1M     & t/o 1.7M    & t/o 2M    & 280k         & 409k        & 30k           & 57k           & 0              & 0              \\ \hline
mot\_ex2 & t/o ---        & t/o ---       & 4.5M         & 6.3M         & t/o 1.7M    & t/o 1.9M    & 358k         & 496k        & 30k           & 57k           & 0              & 0              \\ \hline \hline
wal\_4bit  & 363            & 435           & 283             & 343             & 396            & 479            & 0              & 0             & 300             & 378             & 442            & 486            \\ \hline
wal\_6bit  & 8077           & 9831          & 6887            & 9544            & 11k          & 12k          & 0              & 0             & 8523            & 12k           & 68k          & 54k          \\ \hline
wal\_8bit  & 180k         & 209k        & 177k          & 249k          & 1.2M        & 1.1M        & 0              & 0             & 94k           & 174k          & 5.9M        & 3.8M        \\ \hline
wal\_10bit & t/o ---        & t/o ---       & 2.7M         & 3.7M         & t/o 5.4M    & t/o 2.2M    & 0              & 0             & 249k          & 519k          & t/o 2.8M    & t/o 1.2M    \\ \hline
wal\_12bit & t/o ---        & t/o ---       & t/o 5.2M     & t/o 6M     & t/o 4.1M    & t/o 1.9M    & 0              & 0             & 416k          & 855k          & t/o 2.3M    & t/o 916k     \\ \hline
wal\_14bit & t/o ---        & t/o ---       & t/o 4.9M     & t/o 6.5M     & t/o 3M    & t/o 907k     & 0              & 0             & 500k          & 999k          & t/o 1.2M    & t/o 412k     \\ \hline
wal\_16bit & t/o ---        & t/o ---       & t/o 4.8M     & t/o 6.6M     & t/o 1.9M    & t/o 512k     & 0              & 0             & 941k          & 2M         & t/o 672k     & t/o 196k     \\ \hline
\end{tabular}
\end{sidewaystable}

%%% Local Variables:
%%% mode: latex
%%% TeX-master: "main"
%%% End:

%

\paragraph{\bf Limitations}
%
Although our initial results are promising, our current implementation
has several limitations as well.
%
We have only considered a limited space of low-level multiplier
representations.  Actual representations may include several other
optimizations, e.g., multiplying with constants using bit-shifting
etc.
%
Multiplier operations may also be applied recursively, e.g., the
partial products of a long multiplication may be obtained using
Wallace tree multiplier.
%
While we have noticed significant benefits of adding tautological
assertions encoding the equivalence of pattern-matched terms with
bit-vector multiplication, in general, adding such assertions can hurt
solving time as well.  This can happen if, for example, the assertions
are themselves bit-blasted by the solver, thereby overwhelming the
underlying SAT solver.
%
In addition, the added assertions may be re-written by optimization
passes of the solver, in which case they may not help in identifying
sub-terms representing multiplication in the overall formula.
%
Since the nature of our method is to exploit the potential
structure in the input, we must also adapt all parts of the solver
to be aware of the sought structure as part of our future work.
%
We are currently working to tag the added assertions such that they
are neither simplified in pre-processing nor bit-blasted by the solver.
%
Instead, they should only contribute to the word-level reasoning.
%
Note that our current benchmarks are also limited in the sense that
they do not include examples where multiplication is embedded deep in
a large formula.
%
We are working to make our implementation robust such that
it can reliably work on larger examples, in particular on all the 
SMT-LIB benchmarks.
%
More results in this direction may be found at~\cite{arxiv-version}.


%--------------------- DO NOT ERASE BELOW THIS LINE --------------------------

%%% Local Variables: 
%%% mode: latex
%%% TeX-master: "main"
%%% End:




