

We have implemented our algorithms as a part of $\zthree$~\cite{z3} SMT solver.
%
 We evaluate the performance of our algorithms using benchmarks that are industrial and handcrafted hardware verification problems.
%
We compare our tool with $\zthree$, $\boolector$\cite{boolector} and $\cvcfour$\cite{cvc4}.
%
Our experiments show that the solvers time out on the benchmarks and our tool produces results within the set time limit.

\paragraph{\bf Implementation}
Our implementation of the algorithms in $\zthree$ is 1500 lines long.
%
The algorithms have been implemented in the bit vectors rewrite module of $\zthree$.
%
We chose to work in the rewrite module, since it gives an easy access to the abstract syntax tree of the input formula.
%
An important aspect of the implementation is the ability to exit as early as possible if the match is going to fail.
%
We implemented various preliminary checks \rahul{add some checks}.
%
We have implemented three version of $\ourtool$ by varying
$\textsc{SMTSolver}$.
%
We used  $\zthree$, $\boolector$, and $\cvcfour$ for the variations. 
%
In the case of $\zthree$.
we insert the learnt tautologies in the $\zthree$ solver within the current execution.
%
For $\boolector$ and $\cvcfour$,
we do not run  the $\zthree$ solver after running our matching algorithms,
print the learnt tautologies in a file along with the input formula, and
run the solvers in a separate process on the newly generated formula.
% %
% The mode allows us to run the other solvers 

\paragraph{\bf Benchmarks}
%
Our experiments include 14 benchmarks.
%
Initially, we received an industrial hardware verification benchmark in $\sysver$ involving long multiplication that was not solved by any of the solvers in 24 hours.
%
The example inspired our current work and to evaluate it we generated several similar benchmarks.
%
For long multiplication, we generated benchmarks by varying three characteristics, namely total bit length of the input bit-vectors, width of each block, and assigning specific blocks as equal or set them to zero.
%
Our $\sysver$ benchmarks are fed to STEWord~\cite{Word-level-Symbolic-Trajectory-Evaluation}, a hardware verification tool.
%
STEWord takes $\sysver$ design as input and generates the corresponding SMT1 formula.
%
We convert the SMT1 formula to SMT2 format using $\boolector$.
%
In the process, $\boolector$ extensively simplifies the input formula but retains the overall structure.
%
We have generated benchmarks also for Wallace tree multiplier similar to the long multiplication.
%
For $n$-bit Wallace tree multiplier, we have written a script that takes $n$ as input and generates all the files needed as input by STEWord.
%

\paragraph{\bf Results}
%
We run the solvers on the SMT2 formulas corresponding to the benchmarks by running the solvers on the original input formula and the modified formula after adding the new tautologies.
%

Tables 1-2 specify the results obtained on the benchmarks. We chose timeout to be 600 seconds. All timings are reported in seconds. We use the following versions of the solvers: $\zthree$ - 4.4.2, $\boolector$ - 2.2.0, $\cvcfour$ - 1.4.
%
\begin{table}[t]
\centering
\caption{Long multiplication experiments: Time in seconds}
\label{my-label}
\begin{tabular}{|c|c|c|c|c|c|c|}
\hline
                      & \multicolumn{3}{c|}{\textsc{SMTSolver}}       & \multicolumn{3}{c|}{$\ourtool$}       \\ \hline
Benchmark             & $\zthree$ & $\boolector$ & $\cvcfour$ & $\zthree$ & $\boolector$ & $\cvcfour$ \\ \hline
base                  & 168.51  & 44.50       & 18.37     & 0.39     & 46.02        & 0.02      \\ \hline
ex1\_scaledup         & t/o       & t/o          & t/o        & 1.50     & t/o          & 0.02      \\ \hline
ex2\_scaledup         & t/o       & t/o          & t/o        & 2.57     & t/o          & 0.02      \\ \hline
ex3\_scaledup         & t/o       & t/o          & t/o        & 2.78     & 472.39     & 0.03      \\ \hline
ex5\_scaledup\_2      & 329.56  & 52.06       & 39.89     & 272.39  & 16.46       & 0.02      \\ \hline
mult\_3operands       & t/o       & t/o          & t/o        & t/o       & t/o          & t/o        \\ \hline
sv\_assy              & t/o       & t/o          & t/o        & 0.06     & t/o          & 0.02      \\ \hline
mot\_base\_scaledup   & t/o       & t/o          & t/o        & 11.81    & t/o          & 0.02      \\ \hline
mot\_ex1\_scaledup\_2 & t/o       & t/o          & t/o        & t/o       & 18.43       & 0.02      \\ \hline
mot\_ex2\_scaledup\_2 & t/o       & t/o          & t/o        & t/o       & 15.57       & 0.02      \\ \hline
\end{tabular}
\end{table}

%--------------------- DO NOT ERASE BELOW THIS LINE --------------------------

%%% Local Variables:
%%% mode: latex
%%% TeX-master: "main"
%%% End:



The columns under "SMT Solver" are the time taken by the solvers to prove the satisfiability of the input benchmark. The solvers timout on most of the benchmarks. The benchmarks include two smaller bit sized benchmarks to compare the time taken by the 3 solvers under consideration.

The columns under "Our Solver" is the time taken by the solvers to prove the satisfiability of the input benchmark along with the learnt tautologies. $\zthree$ is run in the first mode, whereas $\boolector$ and $\cvcfour$ are run in the second mode. 
% columns oursolver - time - using backend solvers -Z3, CVC4, Boolector respectively col. numbers  
%
Our experiments indicate that $\cvcfour$ makes best use of the added tautologies. $\cvcfour$ is quickly able to infer that the input formula and the added tautologies are negations of each other justifying the timings captured. 
$\boolector$ and $\zthree$ do not make the above inference, leading to more running time. The added tautologies help $\boolector$ and $\zthree$ reduce the search space, once the sat solver starts solving the bit blasted formula.


\begin{table}[]
\centering
\caption{Wallace tree experiments:Time in seconds}
\label{my-label}
\begin{tabular}{|c|c|c|c|c|c|c|}
\hline
                & \multicolumn{3}{c|}{\textsc{SMTSolver}}       & \multicolumn{3}{c|}{$\ourtool$}       \\ \hline
Benchmark       & $\zthree$ & $\boolector$ & $\cvcfour$ & $\zthree$ & $\boolector$ & $\cvcfour$ \\ \hline
wallace\_4bits  & 0.08     & 0.04        & 0.04      & 0.08     & 0.05        & 0.05      \\ \hline
wallace\_8bits  & 213.33  & 63.80      & 212.79   & 0.44     & 30.60       & t/o        \\ \hline
wallace\_12bits & t/o       & t/o          & t/o        & 1.20     & 227.17     & t/o        \\ \hline
wallace\_16bits & t/o       & t/o          & t/o        & 2.26     & 568.19    & t/o        \\ \hline
\end{tabular}
\end{table}


%%% Local Variables:
%%% mode: latex
%%% TeX-master: "main"
%%% End:


Wallace tree multiplier involves a series of half adders and full adders. As the bit vector size increases, their is a considerable increase in the formula size of the input formula. Despite such increase in formula size, $\zthree$ is quickly able to infer that the input formula and the added tautologies are negations of each other justifying the timings captured. $\boolector$ and $\cvcfour$ do not make the above inference, leading to more running time. This is because the input formula has "and" and "not" gates as its building blocks, whereas $\zthree$ transforms all "ands" to "ors". We have implemented our algorithms in $\zthree$. Therefore, the added tautology has no "ands". This difference in structure caused due to this basic rewrite in $\zthree$, makes it difficult for $\boolector$ and $\cvcfour$ to infer the structural similarity. 

\paragraph{\bf Observations}
%
One benchmark which shows no improvements within the set time limits is mult\_3\_operands. While all the other benchmarks above check for equality of multiplication of two bit vectors, mult\_3\_operands is an attempt to see if the approach works for multiplication of 3 operands which is carried out as two long multiplications. Our observations indicate that none of the solvers are able to infer the structural similarity based on the two tautologies added for the two long multiplications. We carried out similar experiments involving multiple operands, expressions of the form $X*X + Y*Y$, $X*(Y+Z)$, $(X*(Y*Z))*W$. None of the solvers were able to make use of the added tautologies(one for each long multiplication). The problem seems to be on two fronts: one is that to check equality of two terms, the solvers want exact structural similarity and secondly the infrastructure use one assertion to infer the satisfiability of another does not seem to be well developed. 


%--------------------- DO NOT ERASE BELOW THIS LINE --------------------------

%%% Local Variables: 
%%% mode: latex
%%% TeX-master: "main"
%%% End:


