
%--------------------- DO NOT ERASE BELOW THIS LINE --------------------------

%%% Local Variables: 
%%% mode: latex
%%% TeX-master: "main"
%%% End: 

The quest for heuristic strategies for improving the performance of
SMT solvers dates back to the early days of SMT solving.  The list of
papers that describe heuristics for SMT solving is a long and
illustrious one.  Instead of citing these individual papers, we point
the reader to an excellent exposition on this topic
in~\cite{deMoura2013}, which also makes a strong case for developing
languages that enables users to choose their preferred heuristics and
tactics in SMT solvers.

The works that come closest to our work are those developed in the
context of verifying hardware implementations of word-level
arithmetic.  Heuristics for identifying bit-vector (or word-level)
operators from gate-level implementations of these operators have been
developed and optimized by the hardware verification community for
long~\cite{kunz,ciesielski,reveng,earlier-pat-match-synopsys}.  The
use of canonical representations of bit-vector arithmetic operations
have also been explored, as in~\cite{bmd,drechsler}, among others.
These efforts were primarily intended to simplify word-level reasoning
about circuits, and to verify that implementations of complex
arithmetic operators like multiplication indeed implement the
semantics of multiplication.  None of these works were, however,
really targeted at developing SMT solving heuristics for the theory of
fixed-width bit-vectors.

The use of alternative representations of arithmetic operators has
typically been used in SMT solvers when bit-blasting high-level
operators.  For example, {\zthree}~\cite{zthree} uses a specific
Wallace-tree implemention of multiplication when bit-blasting
multiplication operations.  Similar heuristics are also followed in
other solvers like {\boolector}~\cite{boolector} and
{\cvcfour}~\cite{cvcfour}.  However, none of these are intended to
help improve the performance (run-time) of the solver when reasoning
about bit-vector formulas. 
