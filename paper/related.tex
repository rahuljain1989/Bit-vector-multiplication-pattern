The quest for heuristic strategies for improving the performance of
SMT solvers dates back to the early days of SMT solving.  An excellent
exposition on several important early strategies can be found
in~\cite{barrett}.  The importance of orchestrating different
heuristics in a problem-specific manner has been highlighted
in~\cite{deMoura2013}, which also makes a strong case for developing
languages that enables users to choose their preferred heuristics and
tactics for specific problems.

The works that come closest to our work are those developed in the
context of verifying hardware implementations of word-level arithmetic
operations.  There is a long history of heuristics for identifying
bit-vector (or word-level) operators from gate-level implementations
(see, for example,
~\cite{kunz,ciesielski,reveng,earlier-pat-match-synopsys} for a small
sampling).  The use of canonical representations of bit-vector
arithmetic operations have also been explored in the context of
verifying arithmetic circuits like multipliers
(see~\cite{bmd,drechsler}, among others).  However, these
representations usually scale poorly with the bit-width of the
multiplier.  Equivalence checkers determine if two circuits, possibly
designed in different ways, implement the same overall functionality.
State-of-the-art hardware equivalence checking tools, like
Hector~\cite{hector}, make use of sophisticated heuristics like
structural similarities between sub-circuits, complex rewrite rules
and heuristic sequencing of reasoning engines to detect equivalences
between two versions of a circuit.  The rewrite rules used in
Hector~\cite{kolbl} can detect different configurations of circuits
implementing an arithmetic operator and replace them by the
corresponding word-level operator.  Since these efforts are primarily
targeted at establishing the functional equivalence of one circuit
with another, replacing one circuit configuration with another often
works profitably.  However, as argued in Section~\ref{sec:intro}, this
is not always desirable when checking the satisfiability of a formula
obtained from word-level BMC or STE.  Hence, our approach
differs from the use of rewrites used in hardware equivalence
checkers, although there are close parallels between the two. %approaches.

It is interesting to note that alternative representations of
arithmetic operators are internally used in SMT solvers when
bit-blasting high-level arithmetic operators.  However, since an
operator may be implemented in multiple ways, each solver chooses one
(or a few) ways of bit-blasting an operator.  For example,
{\zthree}~\cite{zthree} uses a specific Wallace-tree implementation of
multiplication when blasting multiplication operations.  Since a wide
multiplication operator admits multiple Wallace-tree implementation,
this may not match terms encoding the Wallace-tree implementation of
the same operator in another part of the formula.  Similar heuristics
for bit-blasting arithmetic operators are also used in other solvers
like {\boolector}~\cite{boolector} and {\cvcfour}~\cite{cvcfour}.
However, none of these are intended to help improve the performance
(run-time) of the solver when reasoning about bit-vector formulas.
Instead, they are used to shift the granularity of reasoning from
word-level to bit-level for the sake of completeness, but often
at the price of performance.
% \vspace{-1ex}

%--------------------- DO NOT ERASE BELOW THIS LINE --------------------------

%%% Local Variables: 
%%% mode: latex
%%% TeX-master: "main"
%%% End: 
