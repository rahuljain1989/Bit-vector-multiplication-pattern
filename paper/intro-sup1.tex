% Bit-vector is important
%
In recent years, SMT solving has emerged as a powerful technique for
testing, analysis and verification of hardware and software systems.
A wide variety of tools today use SMT solvers as part of their core
reasoning engines; examples include bounded model
checkers~\cite{hwcbmc,boolector,ebmc,cbmc}, static assertion
checkers~\cite{corral,boogie}, word-level symbolic trajectory
evaluators~\cite{wste}, constrained test
generators~\cite{crv1,crv2,dart}, concolic simulators~\cite{concolic},
among others.  A common approach used by these tools is to model the
behaviour of a system using formulas in a combination of first-order
theories, and reduce the given problem to checking the
(un)satisfiability of a formula in the combined theory.  SMT solvers
play a central role in this scheme of things, since they combine
decision procedures of individual first-order theories to check the
satisfiability of a formula in the combined theory.  Not surprisingly,
techniques to improve the performance of SMT solvers have attracted
significant attention over the years.  The literature contains a rich
body of heuristic strategies for improving the performance of
theory-specific solvers (see~\cite{deMoura2013} for an excellent
exposition).  In this paper, we add to the repertoire of such
heuristics by proposing a pre-processing step that conjoins an input
formula with specially constructed assertions, without changing its
semantics. We focus on formulas in the quantifier-free theory of
fixed-width bit-vectors with multiplication, and show by means of
experiments that our strategy yields significant performance benefits
for three state-of-the-art SMT solvers, namely
{\zthree}~\cite{zthree}, {\cvcfour}~\cite{cvcfour} and
{\boolector}~\cite{boolector}.  Our experiments demonstrate that our
heuristic can achieve reduction in solving times for multiple examples,
by upto several orders of magnitude, especially when the input formula
is unsatisfiable.

%% Theories commonly
%% supported in modern SMT solvers include the quantifier-free theories
%% of fixed-width bit-vectors, arrays, lists, strings, among
%% others~\cite{kroening-book,smtlibv2}.
%% Since reasoning in these individual theories is often much more
%% efficient than reasoning about the bit-level representation of the
%% corresponding data types, techniques based on SMT solving hold a lot
%% of promise as far as scaling to large applications is concerned.  The
%% impressive progress made in SMT solving over the last two
%% decades~\cite{smtprogress} has also substantially lived up to this
%% hope. 

Our primary motivation comes from word-level bounded model checking
(WBMC)~\cite{cbmc,hwcbmc} and word-level symbolic trajectory
evaluation (WSTE)~\cite{wste} of embedded hardware systems.
Specifically, we focus on systems that process data, represented as
fixed-width bit-vectors, using arithmetic operators.  Examples of such
systems include digital signal processing filters, graphics
accelerators, encryption and decryption modules, custom datapath
implementations etc.  When reasoning about these systems, it is often
necessary to check whether a high-level property, specified using
bit-vector arithmetic operators (viz. addition, multiplication,
division), is satisfied by a model of the system implementing a
data-processing algorithm.  For reasons related to performance, power,
area, ease of design etc., complex arithmetic operators with large
bit-widths are often implemented cleverly in embedded systems, using a
composition of several smaller and well-characterized blocks.  For
example, a $128$-bit multiplier may be implemented using one of
several multiplication algorithms, viz. long
multiplication~\cite{long}, Booth-encoded
multiplication~\cite{booth} or Wallace-tree
multiplication~\cite{wallace}, after partitioning its $128$-bit
operands into narrower, say $8$-bit wide, blocks.  SMT formulas
resulting from WBMC/WSTE of such systems are therefore likely to
contain terms with higher-level arithmetic operators (viz. $128$-bit
multiplication) encoding the specification, and terms that encode a
lower-level implementation of these operators in the system (viz. a
Wallace-tree multiplier).  Efficiently reasoning about such formulas
requires exploiting the semantic equivalence of these alternative
representations of arithmetic operators.  Unfortunately, our study,
which focuses on systems using the multiplication operator, reveals
that three state-of-the-art SMT solvers ({\zthree}, {\cvcfour} and
{\boolector}) encounter serious performance bottlenecks in identifying
these equivalences.  This shows up conspicuously when reasoning about
the unsatisfiability of formulas.

\noindent {\bfseries \emph{A motivating example:}} To illustrate
the severity of the problem, we consider the SMT formula arising out
of WSTE applied to a serial multiplier circuit, originally used as a
benchmark in~\cite{wste}.  The circuit reads in two $32$-bit operands
sequentially and stores them in internal registers, before multiplying
them and making the $64$-bit result available in an output register.
The circuit also has several control signals that can be used to
change the flow of control, effectively delaying the computation of
the result.

The property to be checked asserts that if $a$ and $b$ denote the two
operands that are read in, then after the computation is over, the
output register indeed has the product $a * b$, where ``*'' denotes
$32$-bit multiplication.  The RTL design (i.e. system implementation),
as used in~\cite{wste}, makes use of the ``*'' operator in
$\sysver$, effectively specifying a $32$-bit multiplication
operation directly.  The SMT formula resulting from a WSTE run on this
example therefore contains terms with only $32$-bit multiplication
operators, and no terms encoding a lower-level implementation of the
multiplication operator.  This formula is shown to be unsatisfiable
within a fraction of a second by {\boolector} (and also by {\cvcfour}
and {\zthree}).  Note that in WSTE (as also in WBMC), the SMT formula
encodes violation of a property by a bounded run of the systems; hence
unsatisfiability of the formula implies the absence of any bounded
violating runs.

We now change the RTL in the above example to represent the
implementation of $32$-bit multiplication by the long-multiplication
algorithm, where each $32$-bit operand is partitioned into $8$-bit
blocks.  The corresponding WSTE run now yields an SMT formula that
contains terms with $32$-bit multiplication operators (derived from
the property being checked), and also terms that encode the
implementation of $32$-bit multiplication using the
long-multiplication algorithm.  Interestingly, none of {\boolector},
{\cvcfour} and {\zthree} succeeded in deciding the satisfiability of
the resulting formula even after $24$ hours on the same computing
platform.  The heuristic strategies in all the three solvers failed to
identify the semantic equivalence of terms encoding alternative
representations of $32$-bit multiplication, and proceeded
to \emph{bit-blast} the formulas, leading to this spectacular
performance degradation.

\noindent {\bfseries \emph{Problem formulation:}} 
The above example demonstrates that the problem of not being able to
identify equivalence of alternative representations of arithmetic
operators plagues multiple state-of-the-art SMT solvers.  Therefore, a
heuristic that is generic (not solver-specific) would be highly
desirable.  This motivates us to ask: \emph{Can we pre-process an SMT
formula with terms encoding alternative representations of bit-vector
arithmetic operators, in a solver-independent manner, so that multiple
solvers benefit from it?}  We answer this question positively in this
paper for the multiplication operator.  Note, however, that like all
heuristics, we cannot guarantee that all solvers will benefit from our
heuristic on all problem instances.  For example, the problem
described in the motivating example is shown to be unsatisfiable by
{\zthree} in 0.073s and by {\cvcfour} in 0.017s, after application of
our heuristic. Interestingly, {\boolector} didn't benefit from our
heuristic in this case, although there are other examples where
{\boolector} benefits significantly, as shown in
Section~\ref{sec:experiments}.  This is not surprising, since it is
well-known that a heuristic that works well on one problem instance
may not work well for another, even for the same
solver~\cite{deMoura2013}.

\noindent {\bfseries \emph{Some important observations:}} 
At first sight, identifying equivalence between terms encoding
alternative representations of the same arithmetic operator appears to
be one of reverse-engineering an implementation of the operator.  This
problem has been addressed to varying extents by earlier researchers
in different contexts~\cite{reveng,earlier-pat-match-synopsys}.  In the
context of SMT solving, however, complications can potentially arise
if we simply replace a term encoding one representation of an
arithmetic operation by another term encoding a different
representation of the same operation. For example, the same collection
of sums-of-partial-products from long-multiplication can arise from
two different bit-vector multiplication operations, as illustrated in
Example $2$ of long multiplication in Section 2.3.  This makes it
difficult to determine what term replacements should be used to help
the SMT solver decide the satisfiability of the input formula.
Furthermore, even if we are able to uniquely identify the equivalence
of two terms representing the same arithmetic operation, replacing one
term with another may not correlate with improved performance when
checking the (un)satisfiability check of the overall SMT formula.
This can happen for various reasons:
\begin{itemize}
\item Replacing a term by another one is a local ``peep-hole'' transformation
      that is oblivious of the context in which the terms appear in
      the global context of the SMT formula.  What appears beneficial
      locally may indeed hurt the overall (un)satisfiability check.
\item In general, syntactially distinct terms that are semantically equivalent
      may play different roles when reasoning about different parts of
  an SMT formula.  For example, one term may help simplify one sub-formula,
  while the other term may help simplify another sub-formula of the overall
  formula. Replacing one term by another precludes the possibility of both
  terms contributing to improving the performance of the overall satisfiability
  check.
\end{itemize}
Thus, naively replacing a term encoding a specific implementation of a
bit-level arithmetic operator with another term encoding another
implementation of the operator is unlikely to help SMT solving, except
in the simplest of cases.  In this paper, we present a pre-processing
heuristic to address the above problem, focusing only on bit-vector
multiplication.  Given a bit-vector formula $\varphi$ that contains
terms with two different representations of bit-vector multiplication,
our heuristic searches for patterns corresponding to two
multiplication algorithms (long multiplication and Wallace-tree
multiplication) in the terms. Instead of rewriting the matched
sub-terms directly, as has been done in earlier work, we conjoin the
formula $\varphi$ with additional assertions that equate a matched
sub-term with the corresponding bit-vector operator.  Note that each
added assertion is provably a tautology, and does not change the
semantics of the formula.  Significantly, since no re-writes are done,
we can express multiple equivalences in a convenient way.  This is an
important departure from techniques, such as~\cite{kolbl}, that rely
on sophisticated re-writes of the formula. Our experiments show that
the added assertions succeed in preventing bit-blasting in several
cases, while in other cases they significantly help in pruning the
search space even after bit-blasting.  Both benefits eventually
translate to improved performance of the SMT solver.  Since our
heuristic simply pre-processes the input formula by adding assertions,
it is independent of the internals of any specific SMT solver, and can
be combined with multiple solvers.  Indeed, our experiments show that
the performance of different SMT solvers on the pre-processed formula
can vary.  Hence, we propose a portfolio approach to solving the
pre-processed formulas, and show experimentally that a portfolio
solver using pre-processed formulas significantly outperforms a
portfolio solver using the input formulas.



%% Many hardware and software verification problems are translated to the
%% satisfiability of quantifier free bit-vector(QF\_BV)
%% formulas~\cite{hardware,cbmc,more}.
%
