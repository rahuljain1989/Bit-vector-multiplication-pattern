% Bit-vector is important
%
In recent years, SMT solving has emerged as a powerful technique for
testing, analysis and verification of hardware and software systems.
A wide variety of tools today use SMT solvers as part of their core
reasoning
engines~\cite{hwcbmc,boolector,ebmc,cbmc,corral,boogie,crv1,crv2,dart,concolic}.
%; examples include bounded model
%checkers~\cite{hwcbmc,boolector,ebmc,cbmc}, static assertion
%checkers~\cite{corral,boogie}, word-level symbolic trajectory
%evaluators~\cite{wste}, constrained test
%generators~\cite{crv1,crv2,dart}, concolic simulators~\cite{concolic},
%among others.
A common approach used in several of these tools is to model the
behaviour of a system using formulas in a combination of first-order
theories, and reduce the given problem to checking the
(un)satisfiability of a formula in the combined theory.  SMT solvers
play a central role in this approach, since they combine decision
procedures of individual first-order theories to check the
satisfiability of a formula in the combined theory.  Not surprisingly,
techniques to improve the performance of SMT solvers have attracted
significant attention over the years %% .  The literature contains a
%rich
%% body of heuristic strategies for improving the performance of
%% theory-specific solvers
(see~\cite{barrett,deMoura2013} for excellent expositions).  In this
paper, we add to the repertoire of such heuristics by proposing a
pre-processing step that analyzes an input formula, and adds specially
constructed assertions to it, without changing the semantics. We focus
on formulas in the quantifier-free theory of fixed-width bit-vectors
with multiplication, and show by means of experiments that three
state-of-the-art SMT solvers benefit significantly from our heuristic
when solving many benchmarks with multiplication operators.%% , our
%heuristic can significantly reduce the
%% solving time yields significant performance benefits in many cases for
%% namely {\zthree}~\cite{zthree}, {\cvcfour}~\cite{cvcfour} and
%% {\boolector}~\cite{boolector}.
%% %Significantly, our heuristic helps reduce the solving time for
%multiple examples by upto several orders of magnitude when the input
%formula is unsatisfiable.

%% Theories commonly
%% supported in modern SMT solvers include the quantifier-free theories
%% of fixed-width bit-vectors, arrays, lists, strings, among
%% others~\cite{kroening-book,smtlibv2}.
%% Since reasoning in these individual theories is often much more
%% efficient than reasoning about the bit-level representation of the
%% corresponding data types, techniques based on SMT solving hold a lot
%% of promise as far as scaling to large applications is concerned.  The
%% impressive progress made in SMT solving over the last two
%% decades~\cite{smtprogress} has also substantially lived up to this
%% hope. 

The primary motivation for our work comes from word-level bounded
model checking (WBMC)~\cite{cbmc,hwcbmc} and word-level symbolic
trajectory evaluation (WSTE)~\cite{wste} of embedded hardware systems.
Specifically, we focus on systems that process data, represented as
fixed-width bit-vectors, using arithmetic operators.
%% Examples of such
%% systems include digital signal processing filters, graphics
%% accelerators, encryption and decryption modules, custom datapath
%% implementations etc.  
When reasoning about such systems, it is often necessary to check
whether a high-level property, specified using bit-vector arithmetic
operators (viz. addition, multiplication, division), is satisfied by a
model of the system implementing a data-processing algorithm.  For
reasons related to performance, power, area, ease of design etc.,
complex arithmetic operators with large bit-widths are often
implemented by composing several smaller, simpler and
well-characterized blocks.  For example, a $128$-bit multiplier may be
implemented using one of several multiplication
algorithms~\cite{long,booth,wallace}
%viz. long multiplication~\cite{long}, Booth-encoded
%multiplication~\cite{booth} or Wallace-tree
%multiplication~\cite{wallace},
after partitioning its $128$-bit operands into narrower blocks.  SMT
formulas resulting from WBMC/WSTE of such systems are therefore likely
to contain terms with higher-level arithmetic operators
(viz. $128$-bit multiplication) encoding the specification, and terms
that encode a lower-level implementation of these operators in the
system (viz. a Wallace-tree multiplier).  Efficiently reasoning about
such formulas requires exploiting the semantic equivalence of these
alternative representations of arithmetic operators.  Unfortunately,
our study, which focuses on systems using the multiplication operator,
reveals that three state-of-the-art SMT solvers
({\zthree}~\cite{zthree}, {\cvcfour}~\cite{cvcfour} and
{\boolector}~\cite{boolector}) encounter serious performance
bottlenecks in identifying these equivalences.  This manifests
dramatically when reasoning about the unsatisfiability of formulas.

\noindent {\bfseries \emph{A motivating example:}} To illustrate
the severity of the problem, we consider the SMT formula arising out
of WSTE applied to a pipelined serial multiplier circuit, originally
used as a benchmark in~\cite{wste}.  The circuit reads in two $32$-bit
operands sequentially from a single $32$-bit input port, multiplies
them and makes the $64$-bit result available in an output register.
%% The circuit also has
%% several control signals that can be used to change the flow of
%% control, effectively delaying the computation of the result.

The property to be checked asserts that if $a$ and $b$ denote the two
operands that are read in, then after the computation is over, the
output register indeed has the product $a *_{[32]} b$, where
$*_{[32]}$ denotes $32$-bit multiplication.  The system implementation
in~\cite{wste}, described in $\sysver$ (a hardware description
language), makes use of the multiplication operator in $\sysver$
with $32$-bit operands.  The Language Reference Manual of $\sysver$
specifies that this amounts to using a $32$-bit multiplication
operation directly.  The SMT formula resulting from a WSTE run on this
example therefore contains terms with only $32$-bit multiplication
operators, and no terms encoding a lower-level multiplier
implementation.  This formula is shown to be unsatisfiable within a
fraction of a second by {\boolector} (and also by {\cvcfour} and
{\zthree}).%%   Note that in WSTE (as also in WBMC), the SMT formula
%% encodes violation of a property by a bounded run of the system. Hence,
%% unsatisfiability of the formula implies the absence of any bounded
%% violating runs.

We now change the design above to reflect the implementation of
$32$-bit multiplication by the long-multiplication
algorithm~\cite{long}, where each $32$-bit operand is partitioned into
$8$-bit blocks.  The corresponding WSTE run yields an SMT formula that
contains terms with $32$-bit multiplication operator (derived from the
property being checked), and also terms that encode the implementation
of a $32$-bit multiplier using long-multiplication.  Surprisingly,
none of {\boolector}, {\cvcfour} and {\zthree} succeeded in deciding
the satisfiability of the resulting formula even after $24$ hours on
the same computing platform.  The heuristic strategies in these
solvers fail to identify the semantic equivalence of terms encoding
alternative representations of $32$-bit multiplication, and proceed
to bit-blast the formulas, leading to this dramatic run-time blowup.

\noindent {\bfseries \emph{Problem formulation:}} 
The above example demonstrates that the inability to identify semantic
equivalence of alternative representations of arithmetic operators
plagues multiple state-of-the-art SMT solvers.  %% Therefore, a heuristic
%% that helps in this respect and is generic (not solver-specific) would
%% be highly desirable.
This motivates us to ask: \emph{Can we heuristically pre-process an
SMT formula containing terms encoding alternative representations of
bit-vector arithmetic operators, in a solver-independent manner, so
that multiple solvers benefit from it?}  We answer this question
positively for the multiplication operator.  The motivating example,
that originally timed out after $24$ hours on three solvers, is shown
to be unsatisfiable by {\zthree} in 0.073s and by {\cvcfour} in
0.017s, after applying our heuristic. However, {\boolector} does not
benefit from our heuristic on this example.  However, it benefits in
several other examples, as discussed in Section~\ref{sec:experiments}.

\noindent {\bfseries \emph{Term re-writing vs adding tautological assertions:}} 
Prima facie, the above problem can be solved by reverse-engineering a
lower-level representation of a bit-vector arithmetic operator, and by
re-writing terms encoding this representation with terms using the
higher-level bit-vector operator.  Indeed, variants of this approach
have been used in different
contexts~\cite{kunz,ciesielski,kolbl,reveng,earlier-pat-match-synopsys}.
In the context of SMT solving, however, more caution is needed.
%we need to be careful%% complications can potentially
%% arise if we simply
%before re-writing a term encoding one representation of an arithmetic
%operator by another term encoding a different representation.
As shown in Example $2$ of Section~\ref{sec:long-mult}, the same
collection of terms (in this case, sums-of-partial-products) can arise
from two different long-multiplication operations.  This makes it
difficult to decide which of several term re-writes should be
used when there are alternatives. %% to help the SMT solver
%% decide the satisfiability of the input formula.
%Furthermore,
Even if the above dilemma doesn't arise, re-writing one term with
another is a ``peep-hole'' transformation, that may not always
correlate with improved solver performance for various
reasons.  %% Indeed,
%% term re-writing is a ``peep-hole'' transformation that is oblivious of
%% the overall context in which the terms appear in the SMT formula.
%% What appears beneficial locally may not be beneficial in the overall
%% (un)satisfiability check.  In addition,
%% For example, syntactially distinct terms that are semantically
%% equivalent may play different roles when reasoning about different
%% sub-formulas of an SMT formula. 
For example, one term may enable a re-write rule that helps simplify
one sub-formula, while a syntactically distinct but semantically
equivalent term may enable another re-write rule that helps simplify
another sub-formula. Re-writing one term by another precludes the
possibility of both terms contributing to improved performance of
the solver.%satisfiability check.

%% Re-writing a term encoding a low-level representation of an arithmetic
%% operator with another term representing the same operator at a higher
%% level may not always benefit an SMT solver.  
In this paper, we propose a heuristic alternative to term re-writing
when solving bit-vector formulas with multiplication.  Given a
bit-vector formula $\varphi$ containing terms with different
representations of multiplication, our heuristic searches for patterns
in the terms corresponding to two multiplication algorithms, i.e.,
long multiplication and Wallace-tree multiplication. Instead of
re-writing the matched sub-terms with bit-vector multiplication, we
conjoin $\varphi$ with assertions that semantically equate a matched
sub-term with the corresponding multiplication term.  Note that each
added assertion is a tautology, and hence does not change the
semantics of the formula.  Since no re-writes are done, we can express
multiple semantic equivalences without removing any syntactic term
from the formula.  This is an important departure from earlier
techniques, such as~\cite{kolbl}, that rely on sophisticated re-writes
of the formula. Our experiments show that the added tautological
assertions succeed in preventing bit-blasting in several cases, while
in other cases, they help in pruning the search space even after
bit-blasting.  Both effects translate to improved performance of the
SMT solver.  Furthermore, since our heuristic only adds assertions to
the input formula, it is relatively independent of the internals of
any specific solver, and can be used with multiple solvers. %% Our
%experiments show that the
%performance
%% of different SMT solvers on the pre-processed formulas can vary.
%% Hence, we propose a portfolio approach to solving the pre-processed
%% formulas.  We show experimentally that a portfolio solver using
%% pre-processed formulas significantly outperforms a portfolio solver
%% using the original formulas.



%% Many hardware and software verification problems are translated to the
%% satisfiability of quantifier free bit-vector(QF\_BV)
%% formulas~\cite{hardware,cbmc,more}.
%
