%
In this section, we will present our method for solving
formulas that contain implementations of multiplications.
%
Our method first attempts to identify multiplications that
are broken down using long or Wallace tree multiplication.
%
If we suspect some subterms are indeed instances of the
multiplications, we add tautologies stating that the terms are 
equal to the multiplication of the identified bit-vectors.
%
Our identification method may find multiple matches for a subterm.
%
We add a tautology for each match.
%
Let us first present our method of identifying long multiplication. 

% In this section, we will present a class of simplification patterns that
% may simplify the formulas with multipliers 
% and the algorithms to detect the patterns.

\subsection{Identifying long multiplication}
\begin{algorithm}[t]
 \caption{\textsc{MatchLong}($t$)}
 \label{alg:long}
 \begin{algorithmic}[1]
   \Require $t$ : a term in $\qfbv$
   \Ensure $M$ : matched multiplications := $\emptyset$
   \If{ $t == (s_{1k_1} \concat ... \concat s_{11})+...+(s_{pk_p} \concat ... \concat s_{p1})$}
   \State Let $w$ be such that for some $s_{ij} = (0^{w} \concat a) * (0^{w} \concat b)$ and  $len(a) == w$ %:=  minimum bit length of $s_{ij}$
   \State $\Lambda := \lambda i. \emptyset$
   \For{ each $s_{ij}$ }
   \State $o := (\sum_{j' < j} len( s_{ij'}))/w$
   \IIf{ $s_{ij} == 0$ } {\bf continue;}
   \If{ $s_{ij} == (0^{w} \concat a) * (0^{w} \concat b)$  and $len(a) == w$} $\Lambda_o.insert( a * b )$
   \Else~\Return{$\emptyset$}
   \EndIf
   \EndFor
   \State \Return{\textsc{getMultOperands}($\Lambda$,$w$)}
   \EndIf
   % \Else
   \State \Return{$\emptyset$}
 \end{algorithmic}
\end{algorithm}  


%--------------------- DO NOT ERASE BELOW THIS LINE --------------------------

%%% Local Variables:
%%% mode: latex
%%% TeX-master: "main"
%%% End:


In Algorithm~\ref{alg:long}, we present a function $\textsc{MatchLong}$
that takes a $\qfbv$ term $t$ and returns a set of matched multiplications.
%
The algorithm and the subsequent algorithms are written such that as soon
as it becomes clear that no multiplication can be identified then
it returns empty set. 
%
At line 1, we match $t$ with sum of concatenations and if match fails
then clearly $t$ is not a long multiplication.
%
At line 2, we identify the block size $w$ used by the long
multiplication.\ashu{clarify}
%
The loop at line $4$ populates the vector of set of partial products $\Lambda$.
%
$\Lambda_i$ denotes the partial products that are aligned at $i$th block.
%
Each $s_{ij}$ must either be $0$ or a partial product of width $w$.
%
Otherwise, $t$ is declared unmatched at line 8. 
%
At line 5, we compute the alignment $o$ for $s_{ij}$.
%
If $s_{ij}$ happens to be a partial product then it is inserted in
$\Lambda_o$ at line 7.
%
At line 9, we call $\textsc{getMultOperands}$ to identify the operands
of the multiplication from $\Lambda$.

\subsection{Partial products to operands}
\begin{algorithm}[t]
 \caption{\textsc{getMultOperandRec}($Cs$)}
 \label{alg:hb}
 \begin{algorithmic}[1]
   \Ensure $Cs$ : array of multisets of the partial products
   \State $m$ : candidate multiplier 
   \State $n$ : candidate multiplicand
   \State Choose largest $k$ such that $Cs_k\neq \emptyset$
   \If{$Cs_k == \{a*b\} $}
   \State $m_k := a; n_k := b$; $backtrack_k := \lfalse$; 
   \Else
   \State \Return{ FAILED}
   \EndIf
   \State $i := k$
   \While{ $i > 0$}
   \State $i = i - 1 $
   \State $C := Cs_i$
   \For{ $j \in (k-1)..(i+1)$}
   \If{$m_{j} \neq 0$ and $n_{k+i-j} \neq 0$}
   \IIf{ $m_j*m_{k+i-j} \not\in C$} {\bf goto }{\textsc{Backtrack}}
   \State $C := C - \{m_j*m_{k+i-j}\}$
   \EndIf
   \EndFor
   % \IIf{ $|C| > 2$} {\bf goto }{\textsc{Backtrack}}
   \State {\bf match} $C$ {\bf with}
   \State\quad $\mid$ $\{m_k*b,n_k*d\}$ $\rightarrow$ $m_i := d; n_i := b$;
   $backtrack_i := (m_k == n_k)$; 
   \State\quad $\mid$ $\{m_k*n_k\}$ $\rightarrow$ $m_i := 0; n_i := m_k;$
   $backtrack_i := \ltrue$; 
   \State \quad$\mid$ $\{m_k*b\}$ $\rightarrow$ $m_i := 0; n_i := b;$
   $backtrack_i := (m_k == n_k)$; 
   \State \quad $\mid$ $\{n_k*b\}$ $\rightarrow$ $m_i := b; n_i := 0;$\
   $backtrack_i := \lfalse$; 
   \State \quad $\mid$ $\{\}$ $\rightarrow$ $m_i := 0; n_i := 0;$\
   $backtrack_i := \lfalse$; 
   \State \quad $\mid$ $\_$ $\rightarrow$ {\bf goto }{\textsc{Backtrack}};
   \State {\bf continue;}
   \State \textsc{Backtrack:}
   \State \quad Choose smallest $i' > i$ such that $backtrack_i == \ltrue$
   \State \quad {\bf if} no $i'$ found {\bf then} \Return{ FAILED}
   \State \quad $i := i'$
   \State \quad \textsc{SWAP}($m_i,n_i$); $backtrack_{i} := \lfalse$
   \EndWhile
   \State Let $l$ be the smallest number such that $Cs_l\neq \emptyset$
   \State Choose $o \in 0..l$\ashu{More constraints over needed?!!}
   \State Right shift $m$ until $o$ trailing $0$ blocks in $m$
   \State Right shift $m$ until $l-o$ trailing $0$ blocks in $n$
   \State \Return $(m,n)$
 \end{algorithmic}
\end{algorithm}  

%--------------------- DO NOT ERASE BELOW THIS LINE --------------------------

%%% Local Variables:
%%% mode: latex
%%% TeX-master: "main"
%%% End:


In Algorithm~\ref{alg:operand}, we present a function
$\textsc{getMultOperands}$ that takes a $\qfbv$ term $t$ and returns a
set of matched multiplications.
%
At line 1, we compute $h$ and $l$ that establishes the range of search for
operands.
%
We maintain two candidate operands $x$ and $y$ of size $hw$.
%
We also maintain a vector of bits $backtrack$ that encodes 
the possibility of flipping the uncertain decisions.
%
Due to the scheme of the long multiplication, the highest
non-empty entry in $\Lambda$ must be a singleton set.
%
If $\Lambda_h$ contains a single partial product $a*b$,
we assign $x_h$ and $y_h$ the operands of $a*b$ arbitrarily.
%
Otherwise, we declare failure of matching and return $\emptyset$.
%
We set $backtrack_h$ to be $\lfalse$, which states that
no need of backtracking at index $h$.
%
The loop at line 8 iterates using from $h$ to $1$ index $i$.
%
In each iteration, it assigns a value to $x_i$, $y_i$, and $backtrack_i$. 
%

The algorithm may not have enough information at $i$th iteration and
the chosen value for the variables may be wrong.
%
Whenever, the algorithm realizes that such a mistake has happened
it jumps to line 28.
%
It increases back the value of $i$
to the latest $i'$ that allowed backtracking.
%
it swaps the assigned values of $x_i$, $y_i$, and disables future
backtracking at $i$ by setting $backtrack_i$ to
$\lfalse$.
%

Now let us look back again at the loop at line 8.
%
At line 9, we decrement $i$ and $\Lambda_i$ is copied in $C$.
%
At index $i$, the sum of the aligned partial products is the following.
$$
x_{h}*y_{i} + \underbrace{x_{h-1}*y_{i+1} + \dots + x_{i+1}*y_{h-1}}_{\text{operands seen at the earlier iterations}} + x_{i}*y_{h}
$$
We have already chosen the operands of the middle multiplications in the previous iterations.
%
Only the multiplication at the extreme end have $y_i$ and $x_i$ that are
not assigned yet.
%
In the loop at line 10, we remove the middle term and if any of the needed
term is missing then we may have made a mistake earlier and we jump for
backtracking.
%
After the loop, we should have left with at most two partial products in $C$.
%
We match $C$ with six possibilities at lines 14-20 and
update $x_i$, $y_i$, and sets $backtrack_i$ accordingly.
%
In some cases we determine the value of $x_i$ and $y_i$ and the other
cases we are not certain. 
%
In the following list we discuss each of the cases.
%
\begin{itemize}
\item[line 15:] If $C$ has two elements $x_h*b$ and $y_h*d$ then
there is an ambiguity in choosing $x_i$ and $y_i$
if $x_h == y_h$.
%
In the case, we set $backtrack_i$ to $\ltrue$.
\item[line 16:] If $C$ has a single element $x_h*y_h$ then again there  
are two possibilities and we set $backtrack_i$ to $\ltrue$.
\item[line 17:] If $C = \{x_h*b\}$ and $b$ is not $y_h$ then 
  similar to the first case there is an ambiguity in
  choosing $x_i$ and $y_i$ if $x_h == y_h$. Line 18 is similar.
\item[line 19:] If $C$ is empty then certainly $x_i = y_i = 0$.
\item[line 20:] If none of the above patterns match then we jump for
  backtracking.
\end{itemize}
%
At line 21, we check if $i==1$ that means a match has been successful.
%
To find the appropriate operands, we need to right shift $x$ and $y$
such that the total number of their trailing zero blocks is $l-1$.
%
We add the match to the match store $M$.
%
And, the algorithm proceeds for backtracking.

\subsection{Identifying Wallace tree multiplication}
\begin{algorithm}[t]
 \caption{\textsc{RecognizeWallaceTree}($t$)}
 \label{alg:grade}
 \begin{algorithmic}[1]
   \Ensure $t$ : a term in $\qfbv$
   \If{ $t == (t_{k} \concat ... \concat t_{1})$}
   % \State $w$ :=  minimum bit length of $s_{ij}$
   \State $\Lambda := \lambda i. \emptyset$;
   $\Delta := \lambda i. \emptyset$
   \For{ $i \in 1..k$ }
   \State $S$ := $\{t_i\}$
   \For{ $S \neq \emptyset$} 
   \State $t \in S;$ $S := S - \{t\}$
   \If{ $t == s_1 \lxor .... \lxor s_p$}
   \State $S := S \union \{s_1,..,s_p\}$;$\Delta_i := \Delta_i \union \{s_1,..,s_p\}$
   \ElsIf{$t$ is carry output of $fullAdder(a,b,c)$ and $a,b,c \in \Delta_{i-1}$}
   \State $\Delta_i.insert( t )$
   \State $\Delta_{i-1} := \Delta_{i-1}- \{a,b,c\}$
   \ElsIf{$t$ is carry output of $halfAdder(a,b)$ and $a,b \in \Delta_{i-1}$}
   \State $\Delta_i.insert( t )$
   \State $\Delta_{i-1} := \Delta_{i-1}- \{a,b\}$
   \ElsIf{$t == a \land b$}
   \State $\Lambda_i.insert( a * b )$;
   \IIf{ $t \neq t_i$} $\Delta_i.insert( t )$
   \Else~\Return{$\emptyset$}
   \EndIf
   \EndFor
   \IIf{ $i > 1$ and $ \Delta_{i-1} \neq \emptyset $}~\Return{$\emptyset$}
   \EndFor
   \State \Return{\textsc{getMultOperand}($\Lambda$,$1$)}
   \EndIf
   % \Else
   \State~\Return{$\emptyset$}
 \end{algorithmic}
\end{algorithm}  


%--------------------- DO NOT ERASE BELOW THIS LINE --------------------------

%%% Local Variables:
%%% mode: latex
%%% TeX-master: "main"
%%% End:


A Wallace tree has a cluster of adders that take partial products and 
and carry bits as input to produce the output bits.
%
In our matching algorithm, we find the set of inputs
to the adders for an output bit and classify them into
partial products and carry bits.
%
The sum output of a half/full adders are the result of 
xor operations of inputs.
%
To find the input to the adders, we start from a
output bit and follow backward until we find input that
are not the result of $\lxor$.

In Algorithm~\ref{alg:wallace}, we present a function
$\textsc{MatchWallace}$ that takes a $\qfbv$ term $t$ and returns a
set of matched multiplications.
%
At line 1, $t$ is matched with a concatenation of single bit terms $t_1$,..,$t_n$.
%
Similar to Algorithm~\ref{alg:long},
we maintain the partial product store $\Lambda$.
%
For each $i$,
we also maintain the multiset of terms $\Delta_i$ that were used as 
input to the adders for the $i$th bit.
%
In the loop at line 6, we traverse to the subterms until
the subterm is not the result of xors.
%
In the traversal, we also collect the inputs of xors in $\Delta_i$, which
will help us in checking that all the carry inputs in adders for $t_{i+1}$
is generated by the adders for $t_i$.
%
If the term $t$ is not the result of xors then
we have the following possibilities.
%
\begin{itemize}
\item[line 10-13:]
  If $t$ is the carry bit of a half/full adder, we remove the inputs 
  the adder from $\Delta_{i-1}$.
\item[line 14-15:] If $t$ is a partial product, we record this in $\Lambda_i$
\item[line 16:] Otherwise, we return $\emptyset$, which means the match has failed.
\end{itemize}
At line 17, we check that $\Delta_{i-1} = 0$, i.e., all carry bits from
adders for $t_{i-1}$ are consumed by the adders for $t_i$ exactly once.
%
If this check fails then we return $\emptyset$.
%
After the loop at 6, we have collected the partial products in $\Lambda$.
%
At line 18, we call $\textsc{getMultOperands}(\Lambda,1)$ to get all
the matching multiplications.

\subsection{Our solver}
\begin{algorithm}[t]
 \caption{\textsc{OurSolver}($F$)}
 \label{alg:solver}
 \begin{algorithmic}[1]
   \Ensure $F$ : a $\qfbv$ formula
   \State \textsc{SMTSolver}.add($F$)
   \For{ each subterm $t$ in $F$}
   \If{ $M$ := \textsc{RecognizeGradeSchoolMultiplier}($t$) }
   \For{ each $m*n \in M$}
   \State \textsc{SMTSolver}.add($m*n = t$)
   \EndFor
   \EndIf
   \EndFor
   \State \Return{ \textsc{SMTSolver}.checkSat() }
 \end{algorithmic}
\end{algorithm}  

%--------------------- DO NOT ERASE BELOW THIS LINE --------------------------

%%% Local Variables:
%%% mode: latex
%%% TeX-master: "main"
%%% End:


Using the above pattern matching algorithms, we modify an existing
solver $\textsc{SMTSolver}$, which
is presented in the Algorithm~\ref{alg:solver}.
%
$\textsc{OurSolver}$ adds the input formula $F$ in $\textsc{SMTSolver}$.
%
For every subterm of $F$, we attempt to match with both long multiplication
or Wallace tree multiplication.
%
For each discovered matching $x*y$, we add a bit-vector tautology $[x*y = t]$ 
to the solvers, which is obtained after
appropriately zero-padding $x$, $y$, and $t$.
 

\ashu{Correctness proof of the algorithms?}

\section{Proof generation}
\ashu{Are we doing this!}

%--------------------- DO NOT ERASE BELOW THIS LINE --------------------------

%%% Local Variables:
%%% mode: latex
%%% TeX-master: "main"
%%% End:
